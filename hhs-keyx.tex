\documentclass{article}

\usepackage{amssymb, amsmath, amsthm}
\usepackage[utf8]{inputenc}
\usepackage[T1]{fontenc}
\usepackage{unicode}
\usepackage{tikz}
\usepackage{hyperref}
\usepackage[ruled, vlined, linesnumbered]{algorithm2e}
\SetEndCharOfAlgoLine{}

%Shortcuts
\newcommand{\F}{\mathbb{F}}
\newcommand{\Q}{\mathbb{Q}}
\newcommand{\Z}{\mathbb{Z}}
\newcommand{\C}{\mathbb{C}}
\newcommand{\Cl}{\mathcal{C}}
\newcommand{\Graph}{\mathcal{G}}
\renewcommand{\O}{\mathcal{O}}
\newcommand{\softO}{\tilde{O}}
\newcommand{\isom}{\overset{\sim}{\longrightarrow}}
\newcommand{\from}{\ensuremath{\,:\ }}
\newcommand{\set}[1]{\left\{#1\right\}}
\newcommand{\suchthat}{\,|\,}
\newcommand{\algstyle}[1]{\textsc{#1}}
\renewcommand{\v}{\vspace{5mm}}
\renewcommand{\frak}{\mathfrak}

\newtheorem{theorem}{Theorem}[section]
\newtheorem{prop}[theorem]{Proposition}
\theoremstyle{definition}
\newtheorem{prob}[theorem]{Problem}

\DeclareMathOperator{\End}{End}
\DeclareMathOperator{\Ker}{Ker}
\DeclareMathOperator{\Card}{Card}
\DeclareMathOperator{\Ell}{Ell}

\title{Towards practical key exchange from ordinary isogeny graphs}
\author{Luca De Feo, Jean Kieffer, Benjamin Smith}

\begin{document}

\maketitle

\begin{abstract}
\end{abstract}

\textbf{Keywords:} Public-key cryptography, elliptic curve isogenies,
complex multiplication, modular curves.

\section{Introduction}
\label{sec:introduction}

A recent trend in cryptography is the development of quantum-safe
cryptosystems: protocols based on mathematical problems not known to
be solvable in polynomial time by quantum computers. With Shor's
algorithm~\cite{shor1994algorithms} ruling out systems based on
integer factorization or discrete logarithms, NIST has launched a
process to standardize the next generation of \emph{post-quantum}
public key cryptosystems~\cite{NIST2016}. In response, NIST has
received 69 proposals, most of them belonging to the three more
popular post-quantum families: lattice-based systems, code-based
systems, and multivariate systems. Among the youngest and least
explored families, \emph{isogeny-based} cryptography features only one
proposal to the NIST competition: the Supersingular Isogeny Key
Encapsulation (SIKE)~\cite{SIKE}.

SIKE is based upon the key-exchange protocol by Jao and De
Feo~\cite{jao+defeo2011}, known as SIDH, itself inspired by earlier
key-exchange constructions by Couveignes~\cite{cryptoeprint:2006:291}
and Rostovtsev and
Stolbunov~\cite{rostovtsev+stolbunov06,stolbunov-red,Stol}. The
origins of isogeny-based cryptography can be traced back to
Couveignes' seminal work ``Hard Homogeneous Spaces'' (HHS), that went
unpublished for ten years before appearing
in~\cite{cryptoeprint:2006:291}. Couveignes viewed HHS more as a
general framework, encompassing various flavors of Diffie-Hellman-like
systems, rather than as one particular protocol. Among the possible
instantiations of HHS, he suggested a new construction based on the
theory of complex multiplication of elliptic curves. It is however
Rostovtsev and Stolbunov who first gave
in~\cite{rostovtsev+stolbunov06} a concrete and detailed realization
of Couveignes' idea.

Rostovtsev and Stolbunov advertised their system as a potential
post-quantum candidate, leading Childs, Jao and Soukharev to introduce
the first sub-exponential quantum algorithm capable of breaking
it~\cite{childs2014constructing}. Hence, being already slow enough to
be unpractical in a classical security setting, the
Rostovtsev-Stolbunov system became even more unusable in a quantum
security setting. This led Jao and De Feo to create the SIDH
key-exchange protocol~\cite{jao+defeo2011}, which to the present day
is not victim to any sub-exponential attack.

The aim of the present paper is to improve and modernize the
Couveignes-Rostovtsev-Stolbunov construction, borrowing techniques
from SIDH and point-counting algorithms, to the point of making it
usable in a post-quantum setting. Although the final result is far
from being practical, we believe it constitutes progress in the
direction of having a valid isogeny-based alternative to SIDH.
Furthermore, our proposed scheme presents some distinct advantages
over SIDH, that we shall discuss in Section~\ref{todo}. While
preparing this paper we were informed of
% todo: say something on CSIDH

The paper is structured as follows.
% todo

\section{Isogenies and complex multiplication}
\label{sec:math}

We recall here basic facts on isogenies of elliptic curves defined
over finite fields. For an in depth introduction to these concepts, we
refer the reader to~\cite{silverman:elliptic}, and for a general
overview of isogenies and their use in cryptography we
suggest~\cite{defeo2017isogenybased}.

In what follows $\F_q$ is a finite field of characteristic $p$ with
$q$ elements, with an algebraic closure $\bar\F_q$. Let $E$ and $E'$
be elliptic curves defined over $\F_q$, an isogeny $ϕ:E→E'$ is a
non-constant algebraic map, mapping the point at infinity of $E$ to
the point at infinity of $E'$; it is also a group morphism from
$E(\bar\F_q)$ to $E'(\bar\F_q)$~\cite[III.4]{silverman:elliptic}. The
degree of an isogeny is its degree as an algebraic map; we will call
$ℓ$-isogeny an isogeny of degree $ℓ$. The kernel of an isogeny,
denoted by $\ker ϕ$, is the set of points of $E(\bar\F_q)$ that is
sent to the identity point of $E'$.

An isogeny from a curve $E$ to itself is called an
\emph{endomorphism}.  Endomorphisms with the operations of addition
and composition form a ring, called the \emph{endomorphism ring} of
$E$ and denoted by $\End(E)$. Besides scalar multiplications, the
simplest example of endomorphism is the \emph{Frobenius map}
\begin{align*}
  π : E &→ E,\\
  (x,y) &↦ (x^q,y^q).
\end{align*}
As an element of $\End(E)$, it satisfies a quadratic equation
$π^2 + q = tπ$, where the integer $t$, called the \emph{trace} of $π$,
fully determines the order of $E$ as $\#E(\F_q)=q+1-t$. A curve is
called \emph{supersingular} if $p$ divides $t$, \emph{ordinary}
otherwise.

Any isogeny can be factored as a composition of a \emph{separable} and
a \emph{purely inseparable} isogeny. \emph{Purely inseparable}
isogenies have trivial kernel, and degree a power of
$p$. \emph{Separable} isogenies are in one-to-one correspondence with
their kernels: for any finite subgroup $G⊂E$ of order $ℓ$ there is a
unique elliptic curve up to isomorphism, denoted by $E/G$, and a
unique $ℓ$-isogeny $ϕ:E→E/G$ up to composition by the same
isomorphism, such that $\kerϕ=G$. In particular $\deg ϕ=\#\ker ϕ$. All
isogenies of degree coprime to $p$ are separable. A \emph{cyclic
  isogeny} is one such that $\ker ϕ$ is a cyclic group. It follows
that any isogeny of degree greater than $1$ can be factored as a
composition of cyclic isogenies of prime degree.

For any $ℓ$-isogeny $ϕ:E→E'$, there is a unique $ℓ$-isogeny
$\hat{ϕ}:E'→E$ such that the compositions $ϕ∘\hat{ϕ}$ and $\hat{ϕ}∘ϕ$
are equal to the multiplication-by-$ℓ$ maps on $E'$ and $E$
respectively. $\hat{ϕ}$ is called the \emph{dual isogeny} to $ϕ$. This
shows that being \emph{$\ell$-isogenous} is an equivalence
relation. Further, a theorem of Tate states that two curves are
isogenous over $\F_q$ if and only if they have the same number of
points over $\F_q$.


\subsection{Isogeny graphs}

In isogeny-based cryptography one is mostly interested in
\emph{isogeny graphs}, i.e.\ (multi)-graphs which vertices are
elliptic curves up to isomorphism, and which edges are isogenies
between them. The study of isogeny graphs was initiated by
Kohel~\cite{kohel} and continued by many
authors~\cite{Gal,fouquet+morain02,GHS,MiretMSTV06,jao+miller+venkatesan09}.
Because of the dual isogeny theorem, isogeny graphs are typically
considered undirected.

We denote by $E[ℓ]$ the subgroup of $ℓ$-torsion points of
$E(\bar\F_q)$.  If $ℓ$ is coprime to $p$, then $E[ℓ]$ is isomorphic to
$(ℤ/ℓℤ)^2$.  Furthermore, if $ℓ$ is prime $E[ℓ]$ contains exactly
$ℓ+1$ cyclic subgroups of order $ℓ$; it follows that, over $\bar\F_q$,
there are exactly $ℓ+1$ distinct (separable) $ℓ$-isogenies starting
from $E$.  Generically, a connected component of the isogeny graph of
$ℓ$-isogenies over $\bar\F_q$ will be an infinite $(ℓ+1)$-regular
graph; a notable exception is the finite connected component of
\emph{supersingular} curves, used in SIDH and related protocols.

If one restricts to isogenies \emph{defined over $\F_q$}, the picture
becomes more complex.  By definition, an isogeny $ϕ:E→E'$ is
\emph{defined over} $\F_q$ if and only if the Frobenius endomorphism
$π$ stabilizes $\ker ϕ$. If $ϕ$ is cyclic, this is equivalent to
saying that $π$ acts like a scalar on the points of $\ker ϕ$.  Thus,
for any prime $ℓ≠p$, the number of outgoing $ℓ$-isogenies from $E$ is
totally understood by looking at how $π$ acts on $E[ℓ]$. Since $E[ℓ]$
is a $ℤ/ℓℤ$-module of rank $2$, the action of $π$ is represented by a
$2×2$ matrix with entries in $ℤ/ℓℤ$ and characteristic polynomial
$X^2-tX+q\mod ℓ$. We then have four possibilities:
\begin{itemize}
\item[(0)] $π$ has no eigenvalues in $ℤ/ℓℤ$, i.e.\ $X^2-tX+q$ is
  irreducible modulo $ℓ$; then $E$ has no $ℓ$-isogenies.
\item[(1.1)] $π$ has one eigenvalue of (geometric) multiplicity one,
  i.e.\ it is conjugate to a non-diagonal matrix
  $\left(\begin{smallmatrix}λ&*\\0&λ\end{smallmatrix}\right)$; then
  there is one $ℓ$-isogeny from $E$.
\item[(1.2)] $π$ has one eigenvalue of multiplicity two, i.e.\ it acts
  like a scalar matrix
  $\left(\begin{smallmatrix}λ&0\\0&λ\end{smallmatrix}\right)$; then
  there are $ℓ+1$ isogenies of degree $ℓ$ from $E$.
\item[(2)] $π$ has two eigenvalues, i.e.\ it is conjugate to a
  diagonal matrix
  $\left(\begin{smallmatrix}λ&0\\0&μ\end{smallmatrix}\right)$; then
  there are two isogenies from $E$.
\end{itemize}

If we denote by $Δ_π=t^2-4q$ the discriminant of the characteristic
equation of $π$, we see that the cases~(1.x) are only possible if $ℓ$
divides $Δ_π$.  For ordinary curves $Δ_π≠0$, thus only a finite number
of primes $ℓ$ will fall in these cases, and they will be mostly
irrelevant to our cryptosystem. Following the literature on the SEA
point-counting algorithm~\cite{schoof95,todo}, if $ℓ$ falls into
case~(0) it will be called an \emph{Atkin prime}, if it falls into
case~(2) it will be called an \emph{Elkies prime}.

We will chiefly be interested in Elkies primes. Since all curves in
the same isogeny class over $\F_q$ have the same number of points,
they also have the same trace $t$ and discriminant $d_π$.
% todo: do we frown upon bold emphasis?
\textbf{It follows that, if $ℓ$ is an Elkies prime for a curve $E$,
  the connected component of $E$ in the graph of $ℓ$-isogenies is a
  finite $2$-regular graph, i.e.\ a cycle.} In the next subsection we
introduce a group action on this cycle, and determine its size.


\subsection{Complex multiplication}

In this subsection we exclusively focus on ordinary elliptic
curves. If $E$ is an ordinary curve with Frobenius map $π$, its
endomorphism ring $\End(E)$ is isomorphic to an
\emph{order}\footnote{Here, an \emph{order} is a subring which is a
  $ℤ$-module of rank $2$.} in the quadratic imaginary field
$ℚ(\sqrt{Δ_π})$ (see~\cite[III.9]{silverman:elliptic}).  A curve such
that it endomorphism ring is isomorphic to some order $\O$ is said to
have \emph{complex multiplication by $\O$}.  For a detailed treatment
of the theory of complex multiplication,
see~\cite{lang1987elliptic,silverman:advanced}.

Denote by $\O_K$ the ring of integers of $K=ℚ(\sqrt{Δ_π})$, it is its
\emph{maximal order}, i.e.\ it contains any other order of $K$.  Hence
$ℤ[π]⊂\End(E)⊂\O_K$, and there is only a finite number of possible
choices for $\End(E)$: write $Δ_π=d^2Δ_K$, where $Δ_K$ is the
discriminant of $\O_K$, then the index $[\O_K:\End(E)]$ must divide
$d=[\O_K:ℤ[π]]$.

It turns out isogenies allow us to \emph{navigate} the various
orders. If $ϕ:E→E'$ is an $\ell$-isogeny, then one of the following
holds (\cite[Prop.~21]{kohel}):
\begin{itemize}
\item $\End(E) = \End(E')$, then $ϕ$ is said to be
  \emph{horizontal};
\item $[\End(E):\End(E')] = ℓ$, then $ϕ$ is said to be
  \emph{descending};
\item $[\End(E'):\End(E)] = ℓ$, then $ϕ$ is said to be
  \emph{ascending}.
\end{itemize}
Notice that the last two cases can only happen if $ℓ$ divides
$d^2=Δ_π/Δ_K$, and thus correspond to cases (1.x) in the previous
subsection.
% todo: do we frown upon bold emphasis?
\textbf{If $ℓ$ does not divide $Δ_π$, then $ϕ$ is necessarily
  horizontal.}

We now present a group action on the set of all curves having complex
multiplication by a fixed order $\O$; the key exchange protocol of
Section~\ref{sec:keyex} will be built on this action. Let $\frak a$ be
% todo: shall we say "invertible" ideal?
an ideal in $\End(E)≃\O$, and define the
\emph{${\frak a}$-torsion} subgroup of $E$ as
\[
E[\frak a] = \set{P\in E(\bar\F_q) \suchthat \sigma(P) = 0\ 
\text{ for all }\sigma\in\frak a}.
\]
This subgroup is the kernel of an isogeny $\phi_{\frak a}$, defined up
to composition by an isomorphism, hence the codomain of
$\phi_{\frak a}$ is well-defined up to isomorphism.  We denote this
codomain by $\frak a\cdot E$, in other words
$\frak a\cdot E = E/E[\frak a]$.  The isogeny $\phi_{\frak a}$ is
always horizontal (i.e.\ $\End(\frak a \cdot E) = \End(E)$), and its
degree is the \emph{norm} of $\frak a$ as an ideal of $\End(E)$.

Write $\Ell_q(\O)$ for the set of isomorphism classes over $\bar\F_q$
of curves with complex multiplication by $\O$, and assume it is
non-empty. It turns out that the construction above may be extended
into a group action: namely, the group of fractional ideals of
$\End(\O)$ acts on $\Ell_q(\O)$. Furthermore, principal ideals act
trivially, so that the action factors as an action of the \emph{ideal
  class group} $\Cl(\O)$ on $\Ell_q(\O)$.  The main theorem of complex
multiplication states that this action is \emph{simply transitive}. In
other terms, $\Ell_q(\End(E))$ is a \emph{principal homogeneous space}
under the group $\Cl(\End(E))$: if we fix a curve $E$ as base point,
we have a bijection
\[
\begin{aligned}
\Cl(\End(E)) &\to \Ell_k(\End(E)) \\
\text{Ideal class of }\frak a &\mapsto \text{Isomorphism class of }\frak a\cdot E.
\end{aligned}
\]

Let now $ℓ$ be an Elkies prime for $E$, so far we have seen that the
connected component of $E$ in the $ℓ$-isogeny graph is a cycle of
horizontal isogenies. Complex multiplication tells us more. The
restriction of the Frobenius to $E[ℓ]$ has two eigenvalues $λ≠μ$, to
which we associate the prime ideals $\frak a=(π-λ,ℓ)$ and
$\hat{\frak a}=(π-μ,ℓ)$, both of norm $ℓ$. We see then that
$E[\frak a]$ is the eigenspace of $λ$, defining an isogeny
$ϕ_{\frak{a}}$ of degree $ℓ$. Furthermore
$\frak a\hat{\frak a} = \hat{\frak a}\frak a = (ℓ)$, implying that
$\frak a$ and $\hat{\frak a}$ are the inverse of one another in
$\Cl(\End(E))$, thus the isogeny $ϕ_{\hat{\frak a}}:\frak a·E→E$ of
kernel $(\frak a·E)[\hat{\frak a}]$ is the dual of $ϕ_{\frak a}$ (up
to isomorphism). Hence, 
% todo: more bold
\textbf{the eigenvalues $λ$ and $μ$ define two opposite
  \emph{directions} on the isogeny cycle, independently of the
  starting curve}, as shown in Figure~\ref{fig:cycle}.  Finally, the
size of the cycle is the order of $(π-λ,ℓ)$ in $\Cl(\End(E))$, thus
partitioning the set $\Ell_q(\End(E))$ into cycles of equal size.

\begin{figure}[t]
  \centering
  \begin{tikzpicture}
    \def\crater{7}
    \foreach \i in {1,...,\crater} {
      \draw[fill] (360/\crater*\i:2cm) circle (2pt);
      \begin{scope}[shorten >=0.1cm,->]
        \draw[blue] (360/\crater*\i : 1.95cm) -- (360/\crater*\i+360/\crater : 1.95cm);
        \draw[red] (360/\crater*\i+360/\crater : 2.05cm) -- (360/\crater*\i : 2.05cm);
      \end{scope}
      \draw[blue] (360/\crater*\i+180/\crater:1.6cm) node {\small$λ$};
      \draw[red] (360/\crater*\i+180/\crater:2cm) node {\small$μ$};
    }
  \end{tikzpicture}
  \caption{Isogeny cycle for an Elkies prime $ℓ$, with \emph{directions} associated to the Frobenius eigenvalues $λ$ and $μ$.}
  \label{fig:cycle}
\end{figure}

\subsection{Modular curves}

Elliptic curves equipped with $\ell$-isogenies (or, more generally, some 
additional structure) can be naturally identified to \emph{points} on a 
mathematical object called a \emph{modular curve}. Constructing modular curves 
in general is not easy, and we will simply recall here some well-known 
properties. As above, let $\ell$ be a prime distinct from the characteristic of 
$\F_q$. 
Then there exists a smooth curve $X_0(\ell)$ defined over $\F_q$ with the 
following property: there is a bijection
\[
X_0(\ell)(\F_q) \isom \set{
\begin{matrix}
\text{Isomorphism classes over $\bar\F_q$ of pairs $(E, G)$}\\
\text{where $G$ is a subgroup of order $\ell$ defined over $\F_q$.}
\end{matrix}
}
\]

Of course, this is not a mere bijection: it is functorial in $\F_q$, and either
side can be deduced from the other by algebraic means.
In this work, we use modular curves presented as \emph{plane curves} up to 
birational equivalence, i.e. given as an equation of the form
$\Phi(X, Y) = 0.$
In order to find such an equation, it is enough to find two functions $X$ and $Y$
 on the curve that generate its function field. One can obtain an equation for $
X_0(\ell)$ using the two functions
\[
\begin{aligned}
X(E, G) &= j(E), \\
Y(E, G) &= j(E/G)
\end{aligned}
\]
where $j(E)$ denotes the $j$-invariant of the elliptic curve $E$.
They are linked by a polynomial equation
\[
\Phi_\ell(X, Y) = 0
\]
where $\Phi_\ell$ is called the \emph{classical modular polynomial} of level $
\ell$, and has coefficients in $\Z$. The polynomial $\Phi_\ell$ can be
obtained in several ways \cite{};
from a practical point of view, it is enough to
compute it once and store it for further use.

These polynomials quickly become very large, hence one uses different 
functions on $X_0(\ell)$ when $\ell$ grows, giving rise to the 
so-called \emph{Atkin} modular polynomials \cite{}. The classical polynomial $\Phi_\ell$ 
is still arguably the simplest to use, since one only has to solve the 
polynomial equation $\Phi_\ell(j(E), Y) = 0$ to find the $j$-invariants of 
curves linked to $E$ by an $\ell$-isogeny. Here one sees modular
equations can be used in the context of isogeny computations.

In general, the curve $\Phi(X, Y) = 0$ is no longer smooth: for example, double 
points may appear, so that $\F_q$-points on this curve are no longer in bijection 
with geometric data (elliptic curves with some structure up to isomorphism)
as above. Still, plane equations are useful. For example, 
one can show that double points of $X_0(\ell)$ in the coordinates $X, Y$ 
correspond to elliptic curves $E$ that have a nontrivial endomorphism of degree 
$\ell^2$, and this is easily controlled: in particular it cannot be the case
when the discriminant of $\End(E)$ is greater than $4\ell^2$.


\section{The Couveignes--Rostovtsev--Stolbunov key exchange protocol}
\label{sec:keyex}

\subsection{Key exchange from abelian Cayley graphs}

The Couveignes--Rostovtsev--Stolbunov key exchange protocol \cite{}
is an instance of a general
framework of key exchanges using abelian Cayley graphs. This general pattern
is in turn an example of Couveignes' `Hard Homogeneous Spaces' key echange
framework \cite{}.

Let us first describe this general
setting before discussing more specific algorithms used in isogeny computations.
The public data is
\begin{itemize}
\item A finite abelian group $G$;
\item A finite set $S$ of elements of $G$;
\item A finite set $X$ on which $G$ acts simply transitively;
\item A fixed element $x_0\in X$;
\item For each $s\in S$, an integer $M_s\geq 1$.
\end{itemize}
We further ask that the action of $G$ on $X$ be \emph{computable}, in the sense
 that there is an algorithm \algstyle{Step} which, given $s\in S$ and $x\in X$ as
input, computes the element $s\cdot x$. To explain the name, let $\Graph(G, S, X)$
be the oriented graph whose vertices are the elements of $X$, and an edge labelled
by $s\in S$ links $x_1$ to $x_2$ if $s\cdot x_1 = x_2$; then the algorithm
\algstyle{Step} allows us to follow one edge in the graph. The graph $\Graph(G, S, X)$
may be (at least theoretically, if not computationally) identified with the usual
Cayley graph associated with the abelian group $G$ and the set $S$ of generators.
Indeed we will see that the security of such schemes relies on the infeasibility
of this identification.

As usual, two participants Alice and Bob want to build a common secret while
communicating over an insecure channel. They may
achieve this using the public data above as
described in algorithms \ref{alg:keyex}, \ref{alg:key} and \ref{alg:walk}.

\begin{algorithm}
    \caption{\algstyle{KeyExchange}: key exchange using an abelian Cayley graph}
    \label{alg:keyex}
    \KwIn{Public data as above}
    \KwOut{A secret $x_S\in X$ shared by Alice and Bob}
    %		
    Alice does\\
			\quad $K_A \gets$ \algstyle{KeySample}()\;
			\quad $x_A \gets$ \algstyle{Walk}$(K_A, x_0)$\;
			\quad \textbf{publish} $x_A$ \;
		Bob does\\
			\quad $K_B \gets$ \algstyle{KeySample}()\;
			\quad $x_B \gets$ \algstyle{Walk}$(K_B, x_0)$\;
			\quad \textbf{publish} $x_B$ \;
		Alice does\\
			\quad $x_{S, A}$ = \algstyle{Walk}$(K_A, x_B)$\;
		Bob does\\
			\quad $x_{S, B}$ = \algstyle{Walk}$(K_B, x_A)$\;
		\Return $x_S = x_{S, A} = x_{S, B}$
\end{algorithm}

\begin{algorithm}
	\caption{\algstyle{KeySample}: construction of an ephemeral key}
	\label{alg:key}
	\KwIn{Public data as above}
	\KwOut{An ephemeral key}
	%
	\For{$s\in S$}{
		$k_s \overset{R}{\gets} [0, M_s]$ \quad (uniformly at random)
	}
	\Return{$(k_s)_{s\in S}$}
\end{algorithm}

\begin{algorithm}
	\caption{\algstyle{Walk}: a walk in the abelian Cayley graph}
	\label{alg:walk}
	\KwIn{Public data as above; an ephemeral key $K = (k_s)_{s\in S}$; a point $x_I\in X$}
	\KwOut{The element $x_F = (\prod_{s\in S} s^{k_s})\cdot x_I$}
	%
	$x_F\gets x_I$\;
	\For{$s\in S$}{
		\lFor{\(0 \le i < k_s\)}{
			$x_F\gets$ \algstyle{Step}$(s, x_F)$
		}
	}
	\Return $x_F$
\end{algorithm}

As the reader may see, this key exchange framework is based on the computation
of several walks in the graph $\Graph(G, S, X)$. Alice and Bob are allowed to
compute steps in the directions specified by the elements of $S$, and the integers
$M_s$ play the role of the maximal number of steps that will be computed
for each generator. Of course, when both $s$ and $s^{-1}$ appear in $S$, one
of $k_s$ and $k_{s^{-1}}$ should be zero, since the steps would otherwise
annihilate each other.
An important remark is that there is no need to represent
elements of $G$ at all in this protocol, or to compute with them directly,
outside the algorithm \algstyle{Step}.

When designing practical instances of abelian Cayley graph
protocols, the main degrees of liberty are the choice of the generators themselves,
and the bounds $M_s$. The cost of the key exchange is a weighted
sum of the $M_s$, while the key space size is the product of these integers. When
the number of generators is sufficient, there is therefore an exponential gap
between the key space size and the cost of the walks.

Of course, in order to be
cryptographically interesting, the action of the group $G$ on $X$ must in particular
be difficult to \emph{invert}: $x_1, x_2\in X$ being given, finding a combination
$g\in G$ of the generators such that $g\cdot x_1 = x_2$ must be a hard problem.
This is the main reason why this framework cannot usually be used with
Cayley graphs given by multiplication in the group, since division in $G$ is usually
a simple operation. To be more precise, the security of this key exchange relies
on the difficulty of the following Diffie--Hellman-like problems:

\begin{prob}[Group Action Decisional Diffie--Hellman (GADDH) problem] Let $x, y\in X$,
$a, b\in G$. Given $x,\ y,\ a\cdot x$ and $b\cdot x$, decide whether $(ab)\cdot x = y$.
\end{prob}

\begin{prob}[Group Action Computational Diffie--Hellman (GACDH) problem] Let $x\in X$,
$a, b\in G$. Given $x,\ a\cdot x$ and $b\cdot x$, compute $(ab)\cdot x$.
\end{prob}

For a particular family of group actions, we will call GADDH assumption the assumption
that the GADDH problem is computationally infeasible, in the sence that the advantage
given by any polynomial-time algorithm is a negligible function of the security parameter
$\log(\Card(G))$.

\subsection{Presentation of the protocol}

The idea of Rostovtsev and Stolbunov is to instanciate this general framework
using the theory of complex multiplication.
Let us fix an ordinary elliptic curve $E_0$ over a finite field $k$, with
endomorphism ring $\O$. Then the ideal class group of $\O$ will play the role of
the abelian group $G$, and the set $X$ is the set $\Ell_k(\O)$ of elliptic curves
with endomorphism ring $\O$.

We also have to describe the set $S$ of generators used. Let $L$ be a list of
small Elkies primes for the curve $E_0$, as defined in section \ref{sec:math}.
Then for each $\ell\in L$, there exists two ideals in $\O$ of norm $\ell$, which
are inverses of each other. These ideal classes will be the elements of $S$. According to
the discussion in section \ref{sec:math}, we may identify these ideal classes with
pairs $(\ell, v)$, where $v$ is one of the two Frobenius eigenvalues modulo $\ell$.

Now the description of algorithm \algstyle{Step} (algorithm \ref{alg:step}) is a direct consequence of
the previous discussions. The subroutine \algstyle{IsogenyKernel}$(E, \ell, j_1)$
is an algorithm which computes the kernel of an isogeny from $E$ to an elliptic curve
with $j$-invariant $j_1$, and that we will discuss in the following section.
Here, we identify isomorphism classes of curves with
$j$-invariants.

\begin{algorithm}
	\caption{\algstyle{Step}: a step in an ordinary isogeny graph}
	\label{alg:step}
	\KwIn{A generator $(\ell, v)$; an isomorphism class $j_I\in \Ell_k(\O)$; a curve $E_I$
		such that $j(E_I) = j_I$; the classical
		modular polynomial $\Phi_\ell(X, Y)$}
	\KwOut{The isomorphism class $j_F$ given by the action of $(\ell, v)$ on $j_I$}
	%
  $P\gets \Phi_\ell(j_I, Y)$\;
	$j_1, j_2 \gets$ \algstyle{Roots}$(P, k)$\;
	$K_1\gets $ \algstyle{IsogenyKernel}$(E_I, \ell, j_1)$\;
	\lIf{\algstyle{HasFrobeniusEigenvalue}$(K_1, v)$}{\Return $j_1$}
	\lElse{\Return $j_2$}
\end{algorithm}

According to section \ref{sec:math}, this algorithm is valid if the discriminant
of $\O$ is larger than $4\ell^2$.
Note that in algorithm \ref{alg:step}, there is no need to know the endomorphism
ring $\O$, to compute directly with ideals or to be able to compute the set
$\Ell_k(\O)$. In order to use this
protocol, one only has to know which primes are Elkies for the curve $E_0$,
and this is easy once we know the Frobenius characteristic equation on $E_0$.
This is equivalent to computing the cardinality of $E_0$, which is
done in polynomial time using Schoof's algorithm \cite{}.
If we replace $v$ with the other
Frobenius eigenvalue modulo $\ell$ in algorithm \ref{alg:step}, we will compute
a step in the direction specified by the other ideal of norm $\ell$:
we are therefore able to use both of them as generators.

According to section \ref{sec:math}, we may give another version for the
algorithm \algstyle{Step}, described in algorithm \ref{alg:steptors}.

\begin{algorithm}
	\caption{\algstyle{Step2}: another way of computing a step in the isogeny graph}
	\label{alg:steptors}
	\KwIn{A generator $(\ell, v)$; an isomorphism class $j_I\in \Ell_k(\O)$; a curve $E_I$
		such that $j(E_I) = j_I$; the classical
		modular polynomial $\Phi_\ell(X, Y)$}
	\KwOut{The isomorphism class $j_F$ given by the action of $(\ell, v)$ on $j_I$}
	%
	$r\gets$ \algstyle{MultiplicativeOrder}$(v, \F_\ell)$\;
	$k_{ext}\gets$ \algstyle{Extension}$(k, r)$\;
	$P_{ext}\gets$ \algstyle{TorsionPoint}$(E_I, \ell, k_{ext})$\;
	$K\gets$ \algstyle{Subgroup}$(P_{ext})$\;
	$E_F\gets$ \algstyle{Quotient}$(E_I, K)$\;
	\Return $j(E_F)$
\end{algorithm}

Algorithm \ref{alg:steptors} is valid when the Frobenius eigenspace with eigenvalue $v$ is
the only line in $E[\ell](\bar{k})$ whose points are defined over the degree $r$ extension
$k_{ext}$ of $k$. In other words, it is valid when the multiplicative order of $v$ in
$\F_\ell$ is strictly smaller than the multiplicative order of the second eigenvalue.
In any case, only one of the two ideals of norm $\ell$ can be eligible for algorithm
\ref{alg:steptors} as it stands. This disadvantage can be avoided in some cases by
using the \emph{twisted} elliptic curve: while this operation does not change the
isomorphism class of the curve, Frobenius eigenvalues are turned into their opposites,
which may change their multiplicative orders in $\Z/\ell\Z$
and allow walking in the other direction in the graph. For example, if
$p = -1\mod\ell$, and $v = 1$ is a Frobenius eigenvalue for $E$, then the
other eigenvalue is $-1$ and one can use algorithm \algstyle{Step2} in both
directions with $r=1$.

Let us now describe the primitives about isogeny computations used in algorithms
\ref{alg:step} and \ref{alg:steptors}. This will allow us to compare the merits
of these two algorithms, which directly influence the choice of the public
parameters.


\section{Algorithms and parameter selection}
\label{sec:alg}

We keep the notations of sections \ref{sec:math}
and \ref{sec:keyex}. In order to analyse the costs of various parts of the
algorithm, we make complexity estimates in terms of operations in the
base field $k$ of order $q$.
We will use the soft-$O$ notation: a cost $\softO(f(\ell, \log q))$ means
$O(f(\ell, \log q)^{O(1)})$. These complexity estimates will
allow us to discuss the relative costs of algorithms \ref{alg:step} and
\ref{alg:steptors}, and guide us in the later choice of the primes $\ell$
and bounds $M_\ell$ for the number of steps that we will finally propose.

\subsection{Description of the subroutines}

In this paragraph, we turn on to algorithms appearing in algorithms \ref{alg:step}
and \ref{alg:steptors} that are somewhat specific to the Rostovtsev-Stolbunov
scheme; generic isogeny computation as in \algstyle{Quotient} will be discussed
in the following paragraph.
\v

Let us first delve into algorithm \ref{alg:step}.
The reader will agree that the purpose of the subroutines \algstyle{Roots}
and \algstyle{MultiplicativeOrder} is clear. Since $\ell\ll\Card(k)$,
we use the well-known Cantor--Zassenhaus method \cite{} in \algstyle{Roots}:
given a polynomial $P$,
we compute $F = X^q \mod P$ using a fast exponentiation method, and then the polynomial
$\gcd(F - X, P)$ is the product of irreducible factors of degree one in $P$
in $k[X]$. In our setting the modular equation has two roots: therefore we only need
to compute square roots to complete \algstyle{Roots}, which can be
done using the Tonnelli--Shanks algorithm \cite{}. The cost is dominated
by that of computing $F$: when $P$ is a modular equation
of level $\ell$, hence degree $\ell + 1$, this costs $\softO(\ell\log q)$ operations in $k$
using asymptotically fast methods for polynomial arithmetic \cite{}.
As for \algstyle{MultiplicativeOrder},
any version of it will do, including the really naive one consisting in computing
all the powers of $v$ until one is equal to 1.

Algorithm \algstyle{HasFrobeniusEigenvalue}
involves computing the Frobenius on the kernel: more precisely, one computes $(X^q, Y^q)$
where $q$ is the order of $k$ in the quotient ring
\[
k[X, Y]/(K(X),\ Y^2 - X^3 - aX - b)
\]
where $a, b$ are given by the curve equation $y^2 = x^3 + ax + b$. Here, the polynomial
$K(X)$ is the degree $\frac{\ell-1}{2}$ polynomial whose roots are the $x$-coordinates of the elements
in the subgroup of order $\ell$. One then checks whether it is equal to $v$ times the generic
point $(X, Y)$ which is computed using the addition law on the curve. In fact, computing
in the double quotient ring above is not even needed. The two quantities we want to compute,
$(X^q, Y^q)$ and $[v](X, Y)$, have a $Y$ factor only in their second coordinate. Thus one
can only store univariate polynomials in $X$ and multiply by the square of $Y$, that is
$X^3 + aX + b$, at appropriate places in the addition algorithm on the curve. Thus the
computations are really done in the ring $k[X]/K(X)$. An addition of points in this
quotient ring amounts to a constant number of multiplications and additions of degree $\ell$
polynomials, which can be done in $\softO(\ell)$ field operations.
The cost for the scalar multiplication is therefore also
$\softO(\ell)$ using a binary exponentiation process. However, the cost of computing
$(X^q, Y^q)$ is $\softO(\ell\log q)$ as in the Cantor--Zassenhaus algorithm above.

There is one last subroutine in algorithm \ref{alg:step} that is important to discuss,
namely the evaluation (and storage) of the bivariate polynomial $\Phi_\ell$. It is a bivariate
polynomial of degree $\ell + 1$ in both variables, which is not sparse and whose coefficients
grow quickly \cite{}. Therefore in first approximation, even its storage could cost as much as
$\O(\ell^2 \log q)$ when the coefficients are reduced in $k$. However, it is in practice a negligible
part of the algorithm if one uses Atkin modular polynomials
when the classical modular polynomials become too cumbersome to store.
\v

As for algorithm \algstyle{Step2}, the algorithm \algstyle{Torsion} works as follows.
The cardinality $C_{ext}$ of $E(k_{ext})$ is easily computed from $\Card(E(k))$
and the degree of the extension using Hasse's theory \cite{}.
Then, pick a random point $Q_{ext}$ on the curve defined over $k_{ext}$ and compute
\[
P_{ext} = \left[\frac{C_{ext}}{\ell}\right]Q_{ext}
\]
until $P_{ext}$ is nonzero. Then $P_{ext}$ is a correct answer.
In general, very few scalar multiplications have to be computed (usually one),
but they involve points defined over $k_{ext}$. If $r$ denotes the degree of
this field over $k$, then all coordinates are written as polynomials of degree
$r$. The size of $C_{ext}$ is roughly that of $q^r$, so that the amount of $k$-operations
 involved is $\softO(r^2\log q)$ using a binary scalar multiplication mechanism.

Once this torsion point $P_{ext}$ is computed, \algstyle{Subgroup} computes its
multiples $[i]P_{ext}$ for every $1\leq i\leq \frac{\ell - 1}{2}$ and collects
their abscissas. This is done using $\softO(r\ell)$ operations in $k$, as
these points are again defined over $k_{ext}$.

When $r$ is very small (for example $r = 1$), algorithm \algstyle{Step2} is
critically faster than algorithm \algstyle{Step}, since we save a factor $\log q$.
The value of $r$ depends on the Frobenius characteristic equation, so depends on
the cardinality of the initial elliptic curve $E_0$. Therefore, constructing
ordinary elliptic curves with cardinality constraints becomes critical to allowing
better cryptosystem performances.

\subsection{Isogeny computations}

The algorithms \algstyle{IsogenyKernel} and \algstyle{Quotient} appearing in
algorithms \ref{alg:step} and \ref{alg:steptors} are well-known primitives in isogeny
computations \cite{}. We briefly report on these algorithms
for the reader's convenience. 

In general, the case of even-degree isogenies is
more complicated than that of odd-degree isogenies. Since we are only
interested in prime-degree isogenies and the prime 2 is never Elkies, we
will content ourselves with the odd-degree case.

The \algstyle{Quotient} algorithm directly uses Vélu's formulas, which are
summarised in the following theorem \cite{}.
\begin{theorem}[Vélu]
Let $E$ be an elliptic curve in short Weierstrass form
\[
E\ :\ y^2 = x^3 + a x + b
\]
and $\ell$ be an odd prime. Let $K(X)$ be a polynomial representing a subgroup $G$
of order $\ell$ on $E$, in the sense that $K(X)$ is the monic polynomial of degree
$n = \frac{\ell - 1}{2}$ whose roots are the $x$-coordinates of the points in $G$.
We write
\[
K(X) = X^n - \sigma_1 X^{n-1} + \cdots + (-1)^n \sigma_n.
\]
($\sigma_i = 0$ if $i>n$).
Then the curve $E/G$ has (up to isomorphism) the following equation:
\[
E/G\ :\ y^2 = x^3 + a' x + b'
\]
with
\[
\begin{aligned}
a' &= a - 5t,\\
b' &= b - 7w
\end{aligned}
\]
where we define
\[
\begin{aligned}
t &= 6 (\sigma_1^2 - 2\sigma_2) + 2 a n,\\
w &= 10 (\sigma_1^3 - 3 \sigma_1\sigma_2 + 3\sigma_3) + 6 a\sigma_1 + 4bn.
\end{aligned}
\]


\end{theorem}

As one can see, algorithm \algstyle{Quotient} only costs a constant number
of operations in $k$ (not even in $k_{ext}$, as the polynomial $K$ has
coefficients in $k$ although it was built from its roots in $k_{ext}$).

In algorithm~\ref{alg:steptors}, the most costly step is usually the
scalar multiplication. As a consequence, it is interesting in practice
to use other models for curves such as Montgomery curves~\cite{}.
An analogue of Vélu's formulas also exist for this model~\cite{}.

\v

As the reader can see, computing a quotient is easy. On the other hand,
computing the kernel of an isogeny given its domain, degree and image is a
somewhat more difficult problem. Here we concentrate on the large
characteristic case ($p\gg\ell$) and suppose that $\ell$ is odd.
We will follow Elkies' document \cite{}, which studies
this problem in the context of the SEA algorithm for point couting on
elliptic curves. It is not the asymptotically fastest algorithm available
(see for example \cite{}),
but using it does not lead to any significant slowdown in our range of applications.

To describe this algorithm, it is convenient to describe isogenies as
rational fractions of the coordinates $x$ and $y$ on the curve. In fact, these
rational fractions can be written in a simple way \cite{}:

\begin{prop}
Let $\phi$ be an $\ell$-isogeny between two elliptic curves written in
Weierstrass form
\[
E\ :\ y^2 = x^3 + ax + b, \quad E_1\ :\ y^2 = x^3 + a_1x + b_1.
\]
Then there exists polynomials $N$ and $D$ of degrees $\ell$ and $\ell-1$,
and a constant $c\in k^*$, such that $D$ is monic and
\[
\phi(x, y) = \left(\frac{N(x)}{D(x)}, cy\left(\frac{N}{D}\right)'(x)\right).
\]
The polynomial $D$ is the square of the kernel polynomial of $\phi$, as defined
above. Denoting by $\phi_x$ the rational fraction $\frac{N}{D}$,
the following differential equation holds:
\[
c(x^3 + ax + b)\phi_x'^2(x) = \phi_x^3 + a_1\phi_x + b_1,
\]
whence
\begin{equation}
\label{diffeq}
2c(x^3 + ax + b)\phi_x'' + c(3x^2 + a)\phi_x' =
 3\phi_x^2 + a_1.
\end{equation}
\end{prop}

We will call $\phi$ \emph{normalised} if the coefficient $c$ is equal to 1.
This property is not intrinsic: it depends on the particular model chosen
for $E$ and $E_1$.

Elkies' strategy for algorithm \algstyle{IsogenyKernel}$(E, \ell, j(E_1))$ can now be described
as follows:
\begin{itemize}
\item Determine an equation for $E_1$ from the equation of $E$, $\ell$ and
 $j(E_1)$ such that the $\ell$-isogeny $\phi$ is normalised.
\item Solve equation \ref{diffeq} in the world of power series in $x^{-1}$ over $k$, up to
 some precision $N$: that is, write
\[
\frac{N(x)}{D(x)} = x + c_0 + c_1 x^{-1} + \cdots + c_N x^{-N} + O(x^{-N-1}).
\]
\item Recover $D$ and the kernel polynomial from $\phi_x$ using rational
 fraction reconstruction.
\end{itemize}

Since we want to determine a polynomial of degree $\ell-1$, choosing a precision $N = 2\ell$
is sufficient when using standard rational reconstruction algorithms such as Berlekamp--Massey \cite{}.

Let us first describe how equation \ref{diffeq} is solved. One readily
has the following relations between the coefficients $c_k$: for all $k\geq 2$,
\begin{equation}
\label{eq:rec}
 (k-1)(2k+5)c_{k+1} = 3\sum_{i=1}^{k-1}c_i c_{k-i}
	- (2k-1)(k-1)a c_{k-1} - (2k-2)(k-2)b c_{k-2}.
\end{equation}
This relation gives $c_{k+1}$ in terms of the previous coefficients provided $k-1$
and $2k+5$ are invertible.

This recursion is initialised using the following remark: since $\phi$
is normalized,
it is the isogeny given by Vélu's formulas starting from its kernel. Thus
one has
\[
c_0 = 0,\quad c_1 = \frac{a - a_1}{5},\quad c_2 = \frac{b - b_1}{7}.
\]
This leads to an algorithm which costs $O(\ell^2)$ operations in $k$, since each
application of equation \ref{eq:rec} has a linear cost, and one as to compute
$c_k$ for all $k\leq 2\ell$.

Let us now describe how to compute an equation for $E_1$ such that the isogeny
is normalised. One also uses modular polynomials for this task:

\begin{prop}
Let $E\ :\ y^2 = x^3 + ax + b$ be an elliptic curve over $k$
 and $\ell$ be an odd prime number. Let $j = j(E)$ and $j_1$
be a root of the equation
\[
\Phi_\ell(j, Y) = 0
\]
where $\Phi_\ell$ is the classical modular polynomial of level $\ell$.
Define
\begin{equation}
\label{eq:norm}
\begin{aligned}
\lambda &= 18 j \ell \frac{b}{a}\ 
	\frac{\frac{\partial \Phi_\ell}{\partial X}(j, j_1)}
			 {\frac{\partial \Phi_\ell}{\partial Y}(j, j_1)},\\
a_1 &= -\frac{\lambda^2}{48 j_1(j_1 - 1728)},\\
b_1 &= \frac{\lambda^3}{864 j_1^2(j_1 - 1728)}.
\end{aligned}
\end{equation}
Then the elliptic curve $E_1\ :\ y^2 = x^3 + a_1x + b_1$ has $j$-invariant $j_1$,
and there is a normalised isogeny from $E$ to $E_1$.
\end{prop}

These formulas can be proved over the field $\C$ of complex numbers using
modular forms, just as the properties relating $\Phi_\ell$ to $\ell$-isogenies.
They cannot be applied if $j$ is equal to zero, or $j_1$ is equal to 0
or 1728. In practice, it is easy to detect curves that are isogenous
to one of these failure cases, since their cardinality can be predicted
using the theory of complex multiplication \cite{}.
We discard these from the beginning.

\v
To end this section, let us mention that these computations can be extended to
other models for elliptic curves, and also to other types of modular polynomials.
See \cite{} for the use of Atkin modular polynomials in this context, and \cite{}
for an analogue of Vélu's formulas using Montgomery equations for curves.

Several optimisations are available to the algorithm \algstyle{IsogenyKernel}
as sketched here. First, one can avoid the rational reconstruction step and
find the coefficients of the kernel polynomial directly from the $c_k$'s,
and an additional quantity that can be computed using the modular equation.
This is how Elkies originally presented this method. Second, one can avoid
the quadratic step \ref{eq:rec} by performing Newton iterations on power
series that satisfy a related differential equation: see \cite{}.


\subsection{Seeking a good initial curve}
\label{sec:initcurve}

According to section~\ref{sec:alg}, the global performances of our protocol
critically depends of the possibility of using algorithm~\ref{alg:steptors}
instead of algorithm~\ref{alg:step} during walks in the isogeny graph.
In turn, this depends on the ability to generate an initial elliptic curve
$E_0$ whose cardinality satisfies certain congruence conditions modulo
small primes. Even without using algorithm~\ref{alg:steptors}, the scheme is
more efficient when the curve $E_0$ has many small Elkies primes, and this
property also depends on its cardinality.

In the case of ordinary curves, in contrast with supersingular ones,
generating curves with prescribed cardinality is a difficult problem~\cite{todo}.
In this section, we describe how to use the Schoof--Elkies--Atkin (SEA) point counting
algorithm with early abort, combined with the use of certain modular curves,
to construct curves whose cardinality satisfies some of these constraints.
This is faster than choosing curves at random and computing their cardinalities
completely until a convenient one is found, but still does not allow us
to use the full power of algorithm~\ref{alg:steptors}.

The SEA algorithm~\cite{schoof95,todo} is a polynomial-time algorithm for
point counting on elliptic curves over a finite field. In order to compute
$C = \# E(\F_p)$, it computes the value of $C\mod\ell$ modulo a series
of small primes $\ell$: it is a \emph{multimodular algorithm}.
Cryptographers are usually interested in generating elliptic curves of
prime or nearly prime order. In this case, one uses an \emph{early-abort}
mechanism: when $C = 0\mod\ell$ for some $\ell$, the algorithm is aborted
and a new curve is chosen.

Here, the situation is the opposite: we rather want elliptic curves
whose cardinality has lots of small factors. To fix ideas, let us choose
\[
p = 7 \left(\prod_{2\leq\ell\leq 380,\ \ell \text{ prime}} \ell\right) - 1
\]
which is a 512-bit prime. Then, algorithm~\ref{alg:steptors} can be used
for $\ell$-isogenies with $r=1$ in both directions if and only if
$C_0 = \# E_0(\F_p)$ satisfies $C=0\mod\ell$.

We now proceed as follows:
\begin{itemize}
\item Choose a smoothness bound $B$ (we used $B = 13$).
\item Pick elliptic curves at random in $\F_p$, and use the SEA
algorithm, aborting when any $\ell\leq B$ with $C\mod\ell\neq 0$
is found.
\item For each curve which passed the tests above, complete the SEA
algorithm and estimate the key exchange running time using this
curve as a public parameter (see section~\ref{sec:exp}).
\end{itemize}
The ``fastest'' curves are now good candidates for the curve $E_0$.

Considering the efficiency of this procedure, it is important to remark
that very few curves will pass the early-abort tests. The bound $B$ is
chosen to balance the overall cost of the first few tests with that of
the complete SEA algorithm for the curves which pass them. Therefore,
its value is somewhat implementation-dependant. In practice, one also
incorporates the test of existence of a Montgomery model for the
elliptic curve as in~\cite{todo:refMontgomery}.

Since we are looking for curves with smooth cardinalities, another
improvement to this procedure is available: instead of using elliptic
curves uniformly chosen at random, we pick random candidates using
an equation for the modular curve $X_1(N)$~\cite{sutherland2012constructing}.
This way, one guarantees the existence of a rational $N$-torsion point
on the elliptic curve without any further computation. In our
implementation, we used $N = 17$. This idea is used, for example,
in the procedure of selecting elliptic curves in the Elliptic Curve Method
for factoring~\cite{todo:refECM}.

We implemented this research using the Sage computer algebra system.
After 17,000 hours of CPU time, we found the elliptic curve
$
	E : y^2 = x^3 + ax + b
$
over $\F_p$ where the prime $p$ is given above, and
\[
\begin{aligned}
a =\ & 1111646484886395301540457987288261558243572147007270 \\ 
& 1808037226672236659358145861235530764279830938006566 \\
& 253824939581765595814774014481950202045501019370057,\\
b =\ & 7929994832424543515682621940590901949591050994041239 \\
& 8234989651750360146055419691525037257274779805461497 \\ 
& 90133670177545414851226849642510122057033563808087.
\end{aligned}
\]
It admits $\ell$-torsion rational points for each $\ell$ in
\[
  \{3, 5, 7, 11, 13, 17, 103, 523, 821, 947, 1723\}
\]
as well as a Montgomery model, and its cardinality is
\[
\begin{aligned}
& 1203734073820884503438338397822280113709202945127019 \\ 
& 7923071397735408251586670085481138030088461790938201 \\
& 874171652771344144043268298219947026188471598838060.
\end{aligned}
\]

In the sequel, we will argue that using this curve brings a 128-bit security level
in the classical as well as quantum setting, and we give running times of the whole
scheme in our implementation.

\section{Security}
\label{sec:sec}

In this section, we study the security offered by our key exchange protocol
in the classical as well as quantum setting. To be brief, it relies on
the following assumptions:

\begin{itemize}
\item[(1)] The distribution in the isogeny graph of
end points of random walks (algorithms~\ref{alg:key} and~\ref{alg:walk})
are computationnally undistinguishable from the uniform distribution;
\item[(2)] Finding an isogeny path between two random elliptic curves with the
same cardinality is a difficult problem.
\end{itemize}

%todo: some kind of security proof (reduction to diffie-hellman-like (2) given (1)) ?
We first address assertion (2), which has already been well studied and
is used in the vast majority of isogeny-based cryptosystem, before
assertion (1) which is more specific to our protocol and depends on
several number-theoretic heuristic hypotheses.

\subsection{Finding isogeny paths}

According to section~\ref{sec:math}, any two elliptic curves $E$
and $E'$ defined over $\F_q$ with the same number of points are
linked by an isogeny $\phi: E\to E'$. This isogeny can be
efficiently represented as a path of cyclic prime-degree isogenies.

No polynomial-time algorithm that computes this path for
random $E$, $E'$ is known. In the classical setting, the best
answer at this time is a ``meet-in-the-middle''-type algorithm
given by~\cite{Gal}:

\begin{theorem}
Assuming the Riemann Hypothesis for quadratic fields (and
some other heuristic assumptions),
there is an algorithm which, given as input $E$, $E'$ with the
same number of points $q+1-t$ over $\F_q$, computes an isogeny path
from $E$ to $E'$.

Let $K = \Q(\pi)$ where $\pi^2 - t \pi + q = 0$. Let $h$ be the
class number of $\O_K$, and $c$ be the conductor of $\Z[\pi]$ in
$\O_K$.
\begin{itemize}
\item[(a)] In the worst case, the algorithm requires $O(q^{3/2}\ln q)$
time and $O(q\ln q)$ space.
\item[(b)] If $c$ is $\ln(q)$-smooth, the algorithm requires
$O(q^{1/4}\ln(q)^{13/2})$ time and $O(q^{1/4})$ space.
\item[(c)] If $c$ is $\ln(q)$-smooth and $h$ is small
($h = O(\ln(q)^N)$ for some $N$), then the algorithm runs in
polynomial time.
\end{itemize}
\end{theorem}

This algorithm may be improved to have only polynomial space
requirement~\cite{GHS, galbraith+stolbunov11}.

The worst case should be discarded in our setting for two reasons.
First, no reasonable method seems to exist to generate elliptic curves
whose conductor has a large prime factor. Second, it is easy to see that
an $\softO(q^{1/4})$ algorithm exists, even if $c$ has large prime factors,
provided that $E$ and $E'$ have the same endomorphism ring: one simply
skips Stage 1 of~\cite{Gal}.
Therefore, in order to avoid polynomial-time attacks, it is important
to ensure that the class number $h$ has exponential size. In fact, the
time complexity of this attack is $\softO(\sqrt{h})$.


In the quantum setting, a subexponential algorithm exists as mentioned
in the introduction. The following result is taken
from~\cite{childs2014constructing}:

\begin{theorem}
Assuming the Generalized Riemann Hypothesis (GRH), there is
a quantum algorithm which, given the same input as above,
computes an isogeny path from $E$ to $E'$ using either
\begin{itemize}
\item $O(L_q(\frac{1}{2},\frac{1}{\sqrt{2}}))$ time and
superpolynomial space,
\item $O(L_q(\frac{1}{2},\frac{3}{\sqrt{2}}))$ time and
polynomial space.
\end{itemize}
\end{theorem}

Although this subexponential attacks exists, we argue that
it brings no improvement over the classical one at the
128-bit security level. Indeed, superpolynomial quantum space
is not likely to be feasible~\cite{todo:qubits}, and for $q\simeq 2^{512}$
we have $q^{1/4} \ll L_q(\frac{1}{2}, \frac{3}{\sqrt{2}}).$

\subsection{The distribution of random walks}

An important aspect of isogeny graphs, used in most
isogeny-based cryptosystem, is their \emph{expansion properties}.
Random walks in these graphs mix rapidly in the sense that they
quickly become undistinguishable from the uniform distribution.
Showing this result currently requires using number-theoretic
assumptions such as GRH.
We quote the following result from~\cite{jao+miller+venkatesan09}:

\begin{theorem}
Fix an elliptic curve $E$ over $\F_q$ and consider the graph $\Graph$
whose vertices are isomorphism classes of elliptic curves over $\F_q$
with endomorphism ring $\End(E)$, and two vertices are joined by an edge
if there exists an isogeny of degree at most $\log(4q)^B$ between them.

Assuming GRH, $\Graph$ is an expander graph in the following sense:
there is a constant $C$, depending only on $B$, such that any random
walk of length at least
\[
	C \frac{\log |\Graph|}{\log\log q}
\]
starting from any vertex lands in any finite subset $S$ of vertices
with probability at least $\frac{\#S}{2|\Graph|}$.
\end{theorem}

This result is encouraging, but several drawbacks prevent us from
using it directly in our setting. First, it does not guarantee
computational indistinguishability between the output of random
walks and the uniform distribution, although lengthening the
random walks along the lines of~\cite[Theorem ...]{todo:expanders}
would be sufficient. Second, the walks we consider are not fully random
walks, because smaller degree isogenies are used more frequently
(as well as degrees for which we apply algorithm~\ref{alg:steptors}),
and the number of steps is bounded for each degree. Most
importantly, this result is likely too weak: in~\cite[7.2]{jao+miller+venkatesan09},
it is asked whether an analogue of the previous theorem would
be true with the sole constraint $B>1$.

Unfortunately, these adaptations seem out of reach even under
GRH, and heuristic assumptions have to be used. Here, we propose using
a very aggressive one: \emph{as long as the key space is large
enough, the output from walks in the isogeny graph using algorithms~\ref{alg:key}
and~\ref{alg:walk} are computationnally undistinguishable from the uniform
distribution}. More precisely, if $L$ is the list of isogeny degrees used
in the key exchange, and $M_\ell$ is a bound for the number of $\ell$-isogenies
we want to compute for each $\ell\in L$, provided $L$ is not too small
and $\prod (2 M_\ell + 1) \geq |\Graph|$, we suppose that
the output is computationnally
undistinguishable from the uniform distribution in $\Graph$.

\subsection{Security of the proposed parameters}

We now argue that the initial curve $E$ we proposed in section~\ref{sec:initcurve}
offers approximately 128-bit security in the classical as well as quantum
setting. For this curve, the discriminant $\Delta_\pi$ of $\Z[\pi]$ is
given by the following prime factorization:
\[
\begin{aligned}
& -2^5 * 20507 * 67429 * 11718238170290677 * 12248034502305872059 \\
& * 60884358188204745129468762751254728712569\\
& * 68495197685926430905162211241300486171895491480444062860794276603493,
\end{aligned}
\]
hence the fundamental discriminant of $\End(E)$ is a 511-bit integer.
According to approximate formulas like~\cite{todo:classnumbers},
the class number $h$ of the maximal order containing $\End(E)$ should
be at the very least a 250-bit integer. Unfortunately,
no currently available algorithm can compute class numbers for such values
in reasonable time. As seen above, this value guarantees security
in both the classical and quantum settings once the walks in
the graph are well distributed, which we admit.

From its construction, this elliptic curve has lots of rational
points of small order. This raises the following question: does
this property have a negative impact on the security of our scheme ?
We believe that the answer is negative, although further study
is necessary for validation.
On the other hand, the structure of the class group of $\End(E)$
\emph{does} have an impact on the security, since it determines the
shape of the isogeny graph and the speed of mixing of random walks.
However, there seems to be no sign that this particular curve
might be weaker than the average.

\section{Experimental results}
\label{sec:exp}

In order to demonstrate that our key-exchange protocol is usable
at standard security levels, we implemented it in the Julia
programming language with the help of the computer algebra package
Nemo~\cite{todo:Nemo}. In this occasion, we developed a small
Julia package built upon Nemo that handles basic functionalities
regarding elliptic curves over finite fields~\cite{todo:package}.

We use the public parameters as given in
section~\ref{sec:initcurve}.
In this unoptimized implementation, we obtain
the following times to perform a scalar multiplication over $\F_{q^r}$:
%todo: on which gear ?

\begin{center}
\begin{tabular}{c|ccccccc}
$r$ & 1 & 3 & 4 & 5 & 7 & 8 & 9 \\
\hline
$t\ (s)$ & 0.02 & 0.10 & 0.15 & 0.24 & 0.8 & 1.15 & 1.3
\end{tabular}
\end{center}

and the approximate time needed to factor the modular equation of level $\ell$
is $0.017 * \ell$ seconds.

Using such data, one can propose a maximal number of steps to be
computed for each prime. In the upper table of figure~\ref{fig:steps}, we
list the primes for which algorithm~\ref{alg:steptors} is used, with
the corresponding extension degree $r$, and a star if the twisted curve
allows to use both directions in the isogeny graph. In the lower table,
we list other Elkies primes for which we apply algorithm~\ref{alg:step}.

\begin{figure}
\centering
\begin{tabular}{cclc}
$r$ & $M_\ell$ & Primes \\
\hline
1* & 409 & \{3, 5, 7, 11, 13, 17, 103\} \\
1 & 409 & \{523, 821, 947, 1723\} \\
3 & 81 & \{19, 661\} \\
4 & 54 & \{1013, 1181\} \\
5 & 34 & \{31, 61, 1321\} \\
7 & 10 & \{547\} \\
8 & 7 & \{881\} \\
9 & 6 & \{1693\}
\end{tabular}
\vfill
\begin{tabular}{cl}
$M_\ell$ & Primes \\
\hline
20 & \{23\} \\
16 & \{29\} \\
12 & \{37\} \\
11 & \{41\} \\
10 & \{43\} \\
9 & \{47\} \\
6 & \{71,73\} \\
5 & \{89\} \\
4 & \{107,109,113\} \\
3 & \{131,151\} \\
2 & \{157,163,167,191,
  193,
  197,
  223,
  229\} \\
1 & \{241,
  251,
  257,
  277,
  283,
  293,
  307,
  317,
  349,
  359,
  373,
	\\ &
  383,
  401,
  421,
  431,
  433,
  439,
  443,
  449,
  457,
  467\} \\
\end{tabular}
\caption{\label{fig:steps} A proposal for the bounds $M_\ell$}
\end{figure}

These numbers were found using the following procedure: given a maximum time
$T$ that one is willing to spend for each prime, one can compute the
associated key space size using the timing estimations above. Simply keep
increasing $T$ until a sufficiently big key space size is reached.
Using this data, we estimate the cost of a single walk in the
isogeny graph to be approximately ... seconds, bringing the global
key-exchange cost to ... seconds. We stress the fact that this implementation
is \emph{not} optimised: in particular, using an optimised C library
for the computation of $X^q$ modulo a polynomial could instantly bring a
gain by a factor at least 2.


\section{Conclusion}

- A quel point a-t-on accéléré le cryptosystème ?
- Ce protocole a-t-il finalement un intérêt dans le cadre de la crypto post-quantique ?


\bibliographystyle{plain}
\bibliography{refs}

\end{document}

%  LocalWords:  Rostovtsev Stolbunov isogenies morphism isogeny
%  LocalWords:  isomorphism coprime isogenous endomorphism bijection
%  LocalWords:  Endomorphisms cryptosystem Elkies Frobenius
