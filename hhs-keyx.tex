\documentclass{article}

\usepackage{amssymb,amsmath}
\usepackage[utf8]{inputenc}
\usepackage[T1]{fontenc}
\usepackage{tikz}

\title{Towards practical key exchange from ordinary isogeny graphs}
\author{Luca De Feo, Jean Kieffer, Benjamin Smith}

\begin{document}

\maketitle

\begin{abstract}
  
\end{abstract}

\section{Introduction}

C'est un exemple de cryptosystème à base d'isogénies, analogue de SIDH
mais en utilisant des courbes ordinaires (d'où le titre, je ne sais
pas ce que vous en pensez).  La structure des isogénies est
différente, ici on est dans le cadre des 'hard homogeneous spaces'
Citer Couveignes, Rostovtsev-Stolbunov Résumer les caractéristiques
principales du cryptosystème: il n'est pas compétitif en pratique, il
existe une attaque quantique sous-exponentielle (en citant les papiers
adéquats).  Dire pourquoi on s'y intéresse tout de même ? (Cela peut
donner une sécurité post-quantique même en présence d'une attaque
sous-exponentielle, on peut chercher à estimer cette sécurité...?) Par
exemple parce qu'on ne sait pas générer des courbes supersingulières
pour SIDH à partir d'autre chose que le j-invariant 0, qui est cassé ?
-> D'où l'objectif de ce papier: une analyse plus fine de la sécurité
quantique, et des améliorations/accélérations du protocole en
pratique.  Présentation rapide du plan.

\section{Mathematical background}

(Je ne pense pas que l'on puisse faire l'économie d'une section comme
celle-ci... Je pense qu'on peut partir du principe que le lecteur
s'intéresse à l'isogeny-based crypto, et donc connaît ce qu'est une
courbe elliptique et une isogénie. Est-ce que l'on peut partir du
principe que le lecteur sait ce qu'est un groupe de classes ?)

Présenter notamment le concept de courbe/équation modulaire, et la
théorie de la multiplication complexe (action du groupe de classes).
On peut aussi éventuellement parler de volcans d'isogénies: dire que
'génériquement' on a un cratère cyclique et c'est tout, mais dans le
cas où l divise le conducteur on peut avoir d'autres branches
descendantes.

On peut essayer de donner une présentation comme dans le papier:
finalement, un idéal de norme l c'est (l, v) où v est la valeur
propre, et pour calculer l'action on calcule une isogénie et l'on
vérifie que la valeur propre du Frobenius est la bonne.

\section{The Rostovtsev-Stolbunov key-exchange protocol}

Pour présenter l'échange de clés en lui-même, sous forme
d'algotithmes.  Détailler jusqu'au niveau 'calcul du noyau' et 'test
de la valeur propre', que l'on laisse sans plus de discussion pour
l'instant.  Mettre l'accent sur les paramètres à notre disposition: le
corps de base, la courbe initiale, le nombre de pas maximum pour
chaque nombre premier.  Faire la remarque que la taille de l'espace de
clés dépend surtout du nombre de nombres premiers utilisés

(Eventuellement dans une autre section) faire un review rapide des
algorithmes à notre disposition pour le calcul d'isogénies, en faisant
référence aux articles pertinents.  Je pense que l'on peut mettre
l'accent que l'on peut atteindre une complexité quasi-linéaire en l
pour toutes les étapes (cf Sutherland 'on the evaluation of modular
equations', bmss, et on a seulement 2 possibilités à tester pour la
valeur propre).

\section{Algorithms and parameter selection}

Faire une première remarque qu'il est plus efficace d'utiliser des
courbes qui ont plus de petits nombres premiers d'Elkies.

Présenter la deuxième version de l'algorithme de calcul de pas qui
utilise des points rationnels (ou qui vivent dans une petite
extension). Cette deuxième version est plus efficace si la valeur
propre a un plus petit ordre multiplicatif.

Cela amène à vouloir chercher des courbes dont la trace mod l a de
bonnes propriétés pour de petits l. Présenter la stratégie de
recherche SEA--early-abort. Cela vaut-il la peine de parler de la
sélection de candidats à l'aide d'une courbe modulaire, vu le peu que
cela fait gagner à la fin ? Je trouve que c'est intéressant, mais
l'intérêt est plutôt théorique je dirais.

\section{Experimental results}

On présente des paramètres pour viser à 128 bits de sécurité
classique.  Donner le corps de base choisi, et la meilleure courbe
obtenue par la recherche ci-dessus, la trace du Frobenius et le
discriminant de l'ordre maximal (qu'il faudra calculer avec ECM par
exemple, car on peut espérer qu'il n'a pas deux grands facteurs
premiers...), et une minoration explicite du nombre de classes.
Proposer aussi le nombre de pas à effectuer pour chaque nombre premier
d'Elkies.

Je devrais travailler le code pour pouvoir atteindre un échange de clé
complet, et être capable de dire: tant de temps a été dépensé en
calculs de Frobenius, tant de temps en multiplications scalaires, etc.

Indiquer l'existence du paquet Julia en préparation.

(Je remarque d'ailleurs que dans le papier SIDH, on mentionne une
performance de 229 secondes pour l'échange de clés au niveau de 128
bits de sécurité, que je n'atteins pas même avec les meilleures
courbes que j'ai trouvées...)

Faire aussi des commentaires sur la sécurité quantique de ces
paramètres, en fonction de la discussion précédente.  On peut faire
des remarques sur l'implem: dire que l'on utilise une version de BMSS
pour le calcul du noyau, et que l'on peut encore stocker des équations
modulaires pour les niveaux que l'on regarde.

\section{Security}

Deux parties: d'abord la sécurité classique, ensuite la sécurité quantique.

Au niveau classique, on indique l'algorithme de Galbraith (\& Hess \&
Smart, ou \& Stolbunov...). Peut-être mentionner le fait que la
sécurité repose sur des heuristiques, à savoir une bonne répartition
des marches aléatoires dans le graphe, à des niveaux très inférieurs à
ce que peut garantir GRH.  Il reste des questions non résolues: quelle
est l'impact de la structure de groupe ? (Dans ses slides Stolbunov
insiste beaucoup pour avoir un groupe cyclique, je n'en vois pas
vraiment l'intérêt si ce n'est une forme de superstition !)  Est-ce
que la structure supplémentaire des courbes peut être exploitée ?

Au niveau quantique: l'attaque est certes sous-exponentielle, mais on
peut tout de même tenter d'évaluer des paramètres de sécurité. On peut
peut-être s'inspirer des recommandations du NIST ? Quelles ressources
demande l'attaque de Childs et al. relativement à ce que le NIST
envisage ?


\section{Conclusion}

- A quel point a-t-on accéléré le cryptosystème ?
- Ce protocole a-t-il finalement un intérêt dans le cadre de la crypto post-quantique ?


\end{document}
