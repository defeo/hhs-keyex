\documentclass{article}

\usepackage{amssymb,amsmath}
\usepackage[utf8]{inputenc}
\usepackage[T1]{fontenc}
\usepackage{tikz}

%Shortcuts
\newcommand{\F}{\mathbb{F}}
\newcommand{\Q}{\mathbb{Q}}
\newcommand{\Z}{\mathbb{Z}}
\newcommand{\C}{\mathbb{C}}
\newcommand{\Cl}{\mathcal{C}}
\renewcommand{\O}{\mathcal{O}}
\newcommand{\isom}{\overset{\sim}{\longrightarrow}}
\newcommand{\from}{\ensuremath{\,:\ }}
\newcommand{\set}[1]{\left\{#1\right\}}
\newcommand{\suchthat}{\,|\,}

\newtheorem{theorem}{Theorem}

\DeclareMathOperator{\End}{End}
\DeclareMathOperator{\Ker}{Ker}
\DeclareMathOperator{\Card}{Card}
\DeclareMathOperator{\Ell}{Ell}

\title{Towards practical key exchange from ordinary isogeny graphs}
\author{Luca De Feo, Jean Kieffer, Benjamin Smith}

\begin{document}

\maketitle

\begin{abstract}
  
\end{abstract}

\section{Introduction}

C'est un exemple de cryptosystème à base d'isogénies, analogue de SIDH
mais en utilisant des courbes ordinaires (d'où le titre, je ne sais
pas ce que vous en pensez).  La structure des isogénies est
différente, ici on est dans le cadre des 'hard homogeneous spaces'
Citer Couveignes, Rostovtsev-Stolbunov Résumer les caractéristiques
principales du cryptosystème: il n'est pas compétitif en pratique, il
existe une attaque quantique sous-exponentielle (en citant les papiers
adéquats).  Dire pourquoi on s'y intéresse tout de même ? (Cela peut
donner une sécurité post-quantique même en présence d'une attaque
sous-exponentielle, on peut chercher à estimer cette sécurité...?) Par
exemple parce qu'on ne sait pas générer des courbes supersingulières
pour SIDH à partir d'autre chose que le j-invariant 0, qui est cassé ?
-> D'où l'objectif de ce papier: une analyse plus fine de la sécurité
quantique, et des améliorations/accélérations du protocole en
pratique.  Présentation rapide du plan.

\section{Mathematical background}

In this paper, we present a key exchange protocol based on the action of
an abelian group that is related to prime-degree isogenies between ordinary
elliptic curves. In this section, we briefly present standard results that
we will use in the sequel.

\subsection{Isogenies and the Frobenius endomorphism}


We assume the reader to have some familiarity with elliptic curves over 
finite fields and isogenies between them. A well-known reference for these 
concepts is the book of Silverman \cite{Sil1}. We will call 
\emph{$\ell$-isogeny} an isogeny of degree $\ell$, where $\ell$ is a prime.
Here we always assume that $\ell$ is prime to the characteristic;
in particular our $\ell$-isogenies are separable. We will call two isogenies
\emph{isomorphic} if they differ by postcomposition by an isomorphism.

Let $\phi\from E\to E'$ be an $\ell$-isogeny over a finite field $k$. Then 
$\phi$ has a \emph{kernel}, which can be described as the set of $\bar{k}$-
points on $E$ mapped by $\phi$ to the neutral element of $E'$. It satisfies
\[
\Card(\Ker(\phi)) = \ell.
\]
Therefore, the points in $\Ker(\phi)$ are $\ell$-torsion points on the curve 
$E$, that is, we have the inclusion
\[
\Ker(\phi) \subset E[\ell](\bar{k}).
\]

Standard theory tells us that $E[\ell](\bar{k})$ is isomorphic to $(\Z/\ell\Z)^2$
as a group. We may thus view it as a vector space of dimension two over the 
finite field $\Z/\ell\Z$ in which $\Ker(\phi)$ is a subspace of dimension one. 
This correspondence between $\ell$-isogenies and subgroups of order $\ell$ is 
one-to-one: if $G$ is a subgroup of $E(\bar{k})$ of order $\ell$, then there 
exists an isogeny $\phi\from E\to E'$ with kernel $G$, and $\phi$ is uniquely 
determined up to isomorphism. In this case we write $E' = E/G$.

Call an isogeny $\phi$ \emph{$k$-rational} if it can be written as rational 
fractions with coefficients in $k$. One can show that $\phi$ is $k$-rational if 
and only if $\Ker(\phi)$ is.
Rationality is related to the Frobenius endomorphism of the curve $E$, which is 
defined as follows. Let $q$ be the order of $k$, and write $E$ as a plane curve 
with coordinates $x, y$. Then the Frobenius of $E$ is
\[
\begin{aligned}
\pi_E \from E &\to E \\
 (x, y) &\mapsto (x^q, y^q).
\end{aligned}
\]

A point on $E$ is $k$-rational if and only if it is invariant under $\pi_E$, 
and a subgroup of $E(\bar{k})$ is $k$-rational if and only if it is (globally) 
stable under $\pi_E$. Going back to the description of $E[\ell](\bar{k})$ as a 
vector space over $(\Z/\ell\Z)$, we are saying that an $\ell$-isogeny $\phi$ is 
$k$-rational if and only if the line $\Ker(\phi)$ is an eigenspace for the 
endomorphism $\pi_E$ restricted to $E[\ell](\bar{k})$. If $v$ denotes the 
eigenvalue of $\pi_E$ on this subspace, we will say that $\phi$ has \emph{
direction $v$}.


The number of rational isogenies from $E$ to another curve is then related to 
the behaviour of $\pi_E$ as an endomorphism of $E[\ell](\bar{k})$. If it acts 
as a scalar matrix, all $\ell+1$ lines are left stable, so $\ell+1$ rational 
isogenies exist; while if it acts as
\[
\left(
\begin{matrix}
\alpha & * \\
0 & \alpha
\end{matrix}
\right)
\quad
\text{(with $\alpha$ and $*$ nonzero in $\Z/\ell\Z$)}
\]
there is only one stable line, hence only one rational isogeny. In these cases, 
we say that $\pi_E$ has a \emph{double eigenvalue} modulo $\ell$, or that $\ell$
 is \emph{ramified}.

Two other situations can arise: either $\pi_E$ has two distinct eigenvalues 
modulo $\ell$ and there are two rational $\ell$-isogenies starting from $E$, or 
$\pi_E$ has no eigenvalue at all and there are no rational $\ell$-isogenies. 
Following the tradition dating back to the study of Schoof's algorithm for 
point counting, we will call such primes \emph{Elkies} and \emph{Atkin} 
respectively.



\subsection{Ordinary elliptic curves and the theory of complex multiplication}

An elliptic curve $E/k$ is called \emph{ordinary} when its endomorphism ring $
\End(E)$ is an order in a quadratic imaginary field (other curves are called 
\emph{supersingular} and their endomorphism rings are orders in a quaternion 
algebra). Here, an \emph{order} is a subring which is a $\Z$-module of maximal 
rank. 

From now on we will restrict attention to ordinary curves and rational 
isogenies. For example, we will call two curves $E$ and $E'$ \emph{isogenous} 
if there exists a rational isogeny from $E$ to $E'$. This is an equivalence 
relation: according to a theorem of Tate, $E$ and $E'$ are isogenous if and 
only if they have the same number of points over $k$. Hasse's theory then 
implies that the rings $\Z[\pi_E]$ and $\Z[\pi_{E'}]$ are naturally isomorphic (
$\pi_{E'}$ being the image of $\pi_{E}$): indeed we have
\[
\pi_E^2 - t_E\pi_E + q = 0
\]
where $t$ is an integer such that
\[
\Card(E(k)) = p + 1 - t_E.
\]
These rings are of finite index in the endomorphism rings, and as a consequence 
$\End(E)$ and $\End(E')$ can be viewed as orders in the \emph{same} quadratic 
imaginary field $K = \Z[\pi_E]\otimes\Q$. Furthermore, these isomorphisms can 
be made coherent across an isogeny class in the sense that the Frobenius 
endomorphism always corresponds to the same element of $K$.

If $\phi\from E\to E'$ is an $\ell$-isogeny, then the orders $\End(E)$ and $\End
(E')$ are very close to each other. One has the following classification: either
\[
\begin{aligned}[l]
&\End(E) = \End(E'),
&\qquad\text{$\phi$ is then called \emph{horizontal}, or} &\\
&[\End(E):\End(E')] = \ell,
&\qquad\text{$\phi$ is then called \emph{descending}, or} &\\
&[\End(E'):\End(E)] = \ell,
&\qquad\text{$\phi$ is then called \emph{ascending}.} &
\end{aligned}
\]

We now present the group action used in the Rostovtsev--Stolbunov key exchange 
protocol. Let $\frak a$ be an ideal in $\End(E)$. Then we have a natural \emph
{${\frak a}$-torsion} subgroup of $E$ to look at:
\[
E[\frak a](\bar{k}) = \set{P\in E(\bar{k}) \suchthat \sigma(P) = 0\ 
\forall\sigma\in\End(E)}.
\]
This subgroup is the kernel of a rational isogeny $\phi_{\frak a}$ that is well-
defined up to postcomposition by an isomorphism, hence the codomain of $\phi_{
\frak a}$ is well-defined up to isomorphism over $\bar{k}$. We will call this 
codomain $\frak a\cdot E$, in other words
\[
\frak a\cdot E = E/E[\frak a].
\]

The isogeny $\phi_{\frak a}$ is always horizontal, hence we have $\End(\frak a
\cdot E) = \End(E)$, and its degree is the \emph{norm} of $\frak a$ as an ideal 
of $\End(E)$.
Let us call $\Ell_k(\O)$ the set of isomorphism classes over $\bar{k}$ of 
curves whose endomorphism ring is isomorphic to $\O$. It turns out that the 
construction above may be extended into a group action: namely, the group of 
fractional ideals of $\End(E)$ acts on the set $\Ell_k(\End(E))$. Furthermore, 
principal ideals act trivially, so that this action factorises as an action of 
the \emph{ideal class group} $\Cl(\End(E))$ on the set $\Ell_k(\End(E))$.

The main theorem of complex multiplication states that this action is \emph{
simply transitive}; in other terms, $\Ell_k(\End(E))$ is a principal 
homogeneous space under the group $\Cl(\End(E))$. If we fix a curve $E$ as a 
base point, we thus have a bijection
\[
\begin{aligned}
\Cl(\End(E)) &\to \Ell_k(\End(E)) \\
\text{Ideal class of }\frak a &\mapsto \text{Isomorphism class }\frak a\cdot E.
\end{aligned}
\]

Remember that the Frobenius characteristic equation
\[
X^2 - t_E X + q = 0
\]
of discriminant $\Delta_E$ is a defining equation for the field $K = \End(E)
\otimes\Q$. The discriminant of $\End(E)$ divides $\Delta_E$ since $\End(E)$ 
contains the ring $\Z[\pi_E]$. Hence, we can summarize the previous discussion 
in the following way: let $\ell$ be an odd prime, prime to the characteristic. 
Then either
\begin{enumerate}
\item $\Delta_E$ is not a square modulo $\ell$. Then $\ell$ is Atkin, and there 
are no rational $\ell$-isogenies starting from $E$. The prime $\ell$ is inert 
in $K$, so that there are no ideals of norm $\ell$ in $\End(E)$.
\item $\Delta_E$ is nonzero and a square modulo $\ell$. Then $\ell$ is Elkies, 
and there are exactly two $\ell$-isogenies starting from $E$ whose directions 
are the two Frobenius eigenvalues modulo $\ell$. They are horizontal since $\ell
$ is prime to $\Delta_E$. The prime $\ell$ splits in $K$, and there are two 
ideals of norm $\ell$ in $\End(E)$ that are complex conjugates of each other, $
\frak a_\ell$ and $\frak a_\ell^{-1}$. One can show that the action of the 
ideal $\frak a_\ell$ in always given by $\ell$-isogenies in the same direction $
v$, which satisfies
\[
\pi_E = v \mod \frak a_\ell.
\]
Of course, the same is true for $\frak a_\ell^{-1}$. Since the eigenvalues are 
distinct, we may safely identify the ideal $\frak a_\ell$ with the pair $(\ell, 
v)$.
\item $\Delta_E$ is zero modulo $\ell$. Then $\ell$ is ramified. The situation 
is more complicated: there can be either zero or one ideal of norm $\ell$ in $
\End(E)$, depending on its conductor, and there may be non-horizontal 
isogenies. In any case, all these isogenies are associated with the same 
Frobenius eigenvalue.

\end{enumerate}


\subsection{Modular curves}

Elliptic curves equipped with $\ell$-isogenies (or, more generally, some 
additional structure) can be naturally identified to \emph{points} on a 
mathematical object called a \emph{modular curve}. Constructing modular curves 
in general is not easy, and we will simply recall here some well-known 
properties. As above, let $\ell$ be a prime distinct from the characteristic of 
$k$. 
Then there exists a smooth curve $X_0(\ell)$ defined over $k$ with the 
following property: there is a bijection
\[
X_0(\ell)(k) \isom \set{
\begin{matrix}
\text{Isomorphism classes over $\bar{k}$ of pairs $(E, G)$}\\
\text{where $G$ is a subgroup of order $\ell$ defined over $k$.}
\end{matrix}
}
\]

Of course, this is not a mere bijection: it is functorial in $k$, and one side 
can be deduced from the other by algebraic means.
In this work, we use modular curves presented as \emph{plane curves} up to 
birational equivalence, i.e. given as an equation of the form
$\Phi(X, Y) = 0.$
In order to find such an equation, it is enough to find to functions $X$ and $Y$
 on the curve that generate its function field. One can obtain an equation for $
X_0(\ell)$ using the two functions
\[
\begin{aligned}
X(E, G) &= j(E), \\
Y(E, G) &= j(E/G)
\end{aligned}
\]
where $j(E)$ denotes the $j$-invariant of the elliptic curve $E$. Indeed, $X$ 
and $Y$ are well defined and algebraic, and can be shown to generate the 
function field of $X_0(\ell)$. They are linked by a polynomial equation
\[
\Phi_\ell(X, Y) = 0
\]
where $\Phi_\ell$ is called the \emph{classical modular polynomial} of level $
\ell$. The coefficients of $\Phi_\ell$ can be obtained by considering the 
modular curve $\Phi_\ell$ over $\C$, for example.

These polynomials quickly become difficult to compute, hence one uses different 
functions on $X_0(\ell)$ when $\ell$ becomes large, giving rise to the so-
called \emph{Atkin modular polynomials}. The classical polynomial $\Phi_\ell$ 
is still arguably the simplest to use, since one only has to solve the 
polynomial equation $\Phi_\ell(j(E), Y) = 0$ to find the $j$-invariants of 
curves linked to $E$ by an $\ell$-isogeny. Here one sees how equations of 
modular curves can be used in the context of isogeny computations.

In general, the curve $\Phi(X, Y) = 0$ is no longer smooth: for example, double 
points may appear, so that $k$-points on this curve are no longer in bijection 
with geometric data as above. Still, plane equations are useful. For example, 
one can show that double points of $X_0(\ell)$ in the coordinates $X, Y$ 
correspond to elliptic curves $E$ that have a nontrivial endomorphism of degree 
$\ell^2$, and this is easily controlled.


\section{The Rostovtsev--Stolbunov key-exchange protocol}

\subsection{Key exchange from abelian Cayley graphs}

Pour présenter l'échange de clés en lui-même, sous forme
d'algorithmes.  Détailler jusqu'au niveau 'calcul du noyau' et 'test
de la valeur propre', que l'on laisse sans plus de discussion pour
l'instant.  Mettre l'accent sur les paramètres à notre disposition: le
corps de base, la courbe initiale, le nombre de pas maximum pour
chaque nombre premier.  Faire la remarque que la taille de l'espace de
clés dépend surtout du nombre de nombres premiers utilisés

(Eventuellement dans une autre section) faire un review rapide des
algorithmes à notre disposition pour le calcul d'isogénies, en faisant
référence aux articles pertinents.  Je pense que l'on peut mettre
l'accent que l'on peut atteindre une complexité quasi-linéaire en l
pour toutes les étapes (cf Sutherland 'on the evaluation of modular
equations', bmss, et on a seulement 2 possibilités à tester pour la
valeur propre).

\section{Algorithms and parameter selection}

Faire une première remarque qu'il est plus efficace d'utiliser des
courbes qui ont plus de petits nombres premiers d'Elkies.

Présenter la deuxième version de l'algorithme de calcul de pas qui
utilise des points rationnels (ou qui vivent dans une petite
extension). Cette deuxième version est plus efficace si la valeur
propre a un plus petit ordre multiplicatif.

Cela amène à vouloir chercher des courbes dont la trace mod l a de
bonnes propriétés pour de petits l. Présenter la stratégie de
recherche SEA--early-abort. Cela vaut-il la peine de parler de la
sélection de candidats à l'aide d'une courbe modulaire, vu le peu que
cela fait gagner à la fin ? Je trouve que c'est intéressant, mais
l'intérêt est plutôt théorique je dirais.

\section{Experimental results}

On présente des paramètres pour viser à 128 bits de sécurité
classique.  Donner le corps de base choisi, et la meilleure courbe
obtenue par la recherche ci-dessus, la trace du Frobenius et le
discriminant de l'ordre maximal (qu'il faudra calculer avec ECM par
exemple, car on peut espérer qu'il n'a pas deux grands facteurs
premiers...), et une minoration explicite du nombre de classes.
Proposer aussi le nombre de pas à effectuer pour chaque nombre premier
d'Elkies.

Je devrais travailler le code pour pouvoir atteindre un échange de clé
complet, et être capable de dire: tant de temps a été dépensé en
calculs de Frobenius, tant de temps en multiplications scalaires, etc.

Indiquer l'existence du paquet Julia en préparation.

(Je remarque d'ailleurs que dans le papier SIDH, on mentionne une
performance de 229 secondes pour l'échange de clés au niveau de 128
bits de sécurité, que je n'atteins pas même avec les meilleures
courbes que j'ai trouvées...)

Faire aussi des commentaires sur la sécurité quantique de ces
paramètres, en fonction de la discussion précédente.  On peut faire
des remarques sur l'implem: dire que l'on utilise une version de BMSS
pour le calcul du noyau, et que l'on peut encore stocker des équations
modulaires pour les niveaux que l'on regarde.

\section{Security}

Deux parties: d'abord la sécurité classique, ensuite la sécurité quantique.

Au niveau classique, on indique l'algorithme de Galbraith (\& Hess \&
Smart, ou \& Stolbunov...). Peut-être mentionner le fait que la
sécurité repose sur des heuristiques, à savoir une bonne répartition
des marches aléatoires dans le graphe, à des niveaux très inférieurs à
ce que peut garantir GRH.  Il reste des questions non résolues: quelle
est l'impact de la structure de groupe ? (Dans ses slides Stolbunov
insiste beaucoup pour avoir un groupe cyclique, je n'en vois pas
vraiment l'intérêt si ce n'est une forme de superstition !)  Est-ce
que la structure supplémentaire des courbes peut être exploitée ?

Au niveau quantique: l'attaque est certes sous-exponentielle, mais on
peut tout de même tenter d'évaluer des paramètres de sécurité. On peut
peut-être s'inspirer des recommandations du NIST ? Quelles ressources
demande l'attaque de Childs et al. relativement à ce que le NIST
envisage ?


\section{Conclusion}

- A quel point a-t-on accéléré le cryptosystème ?
- Ce protocole a-t-il finalement un intérêt dans le cadre de la crypto post-quantique ?


\end{document}
