\documentclass{article}

\usepackage{amssymb, amsmath, amsthm}
\usepackage[utf8]{inputenc}
\usepackage[T1]{fontenc}
\usepackage{tikz}
\usepackage[ruled, vlined, linesnumbered]{algorithm2e}
\SetEndCharOfAlgoLine{}

%Shortcuts
\newcommand{\F}{\mathbb{F}}
\newcommand{\Q}{\mathbb{Q}}
\newcommand{\Z}{\mathbb{Z}}
\newcommand{\C}{\mathbb{C}}
\newcommand{\Cl}{\mathcal{C}}
\newcommand{\Graph}{\mathcal{G}}
\renewcommand{\O}{\mathcal{O}}
\newcommand{\softO}{\tilde{O}}
\newcommand{\isom}{\overset{\sim}{\longrightarrow}}
\newcommand{\from}{\ensuremath{\,:\ }}
\newcommand{\set}[1]{\left\{#1\right\}}
\newcommand{\suchthat}{\,|\,}
\newcommand{\algnamestyle}[1]{\textsc{#1}}

\newtheorem{theorem}{Theorem}[section]
\theoremstyle{definition}
\newtheorem{prob}[theorem]{Problem}

\DeclareMathOperator{\End}{End}
\DeclareMathOperator{\Ker}{Ker}
\DeclareMathOperator{\Card}{Card}
\DeclareMathOperator{\Ell}{Ell}

\title{Towards practical key exchange from ordinary isogeny graphs}
\author{Luca De Feo, Jean Kieffer, Benjamin Smith}

\begin{document}

\maketitle

\begin{abstract}
  
\end{abstract}

\section{Introduction}

C'est un exemple de cryptosystème à base d'isogénies, analogue de SIDH
mais en utilisant des courbes ordinaires (d'où le titre, je ne sais
pas ce que vous en pensez).  La structure des isogénies est
différente, ici on est dans le cadre des 'hard homogeneous spaces'
Citer Couveignes, Rostovtsev-Stolbunov Résumer les caractéristiques
principales du cryptosystème: il n'est pas compétitif en pratique, il
existe une attaque quantique sous-exponentielle (en citant les papiers
adéquats).  Dire pourquoi on s'y intéresse tout de même ? (Cela peut
donner une sécurité post-quantique même en présence d'une attaque
sous-exponentielle, on peut chercher à estimer cette sécurité...?) Par
exemple parce qu'on ne sait pas générer des courbes supersingulières
pour SIDH à partir d'autre chose que le j-invariant 0, qui est cassé ?
-> D'où l'objectif de ce papier: une analyse plus fine de la sécurité
quantique, et des améliorations/accélérations du protocole en
pratique.  Présentation rapide du plan.

\section{Mathematical background}
\label{sec:math}

In this paper, we present a key exchange protocol based on the action of
an abelian group that is related to prime-degree isogenies between ordinary
elliptic curves. In this section, we briefly present standard results that
we will use in the sequel.

\subsection{Isogenies and the Frobenius endomorphism}


We assume the reader to have some familiarity with elliptic curves over 
finite fields and isogenies between them. A well-known reference for these 
concepts is the book of Silverman \cite{Sil1}. We will call 
\emph{$\ell$-isogeny} an isogeny of degree $\ell$, where $\ell$ is a prime.
We always assume that $\ell$ is prime to the characteristic:
in particular our $\ell$-isogenies are separable. We will call two isogenies
\emph{isomorphic} if they differ by postcomposition by an isomorphism.

Let $\phi\from E\to E'$ be an $\ell$-isogeny over a finite field $k$. Then 
$\phi$ has a \emph{kernel}, which can be described as the set of $\bar{k}$-
points on $E$ mapped by $\phi$ to the neutral element of $E'$. It satisfies
\[
\Card(\Ker(\phi)) = \ell.
\]
Therefore, the points in $\Ker(\phi)$ are $\ell$-torsion points on the curve 
$E$, that is, we have the inclusion
\[
\Ker(\phi) \subset E[\ell](\bar{k}).
\]

Standard theory tells us that $E[\ell](\bar{k})$ is isomorphic to $(\Z/\ell\Z)^2$
as a group. We may thus view it as a vector space of dimension two over the 
finite field $\Z/\ell\Z$ in which $\Ker(\phi)$ is a subspace of dimension one. 
This correspondence between $\ell$-isogenies and subgroups of order $\ell$ is 
one-to-one: if $G$ is a subgroup of $E(\bar{k})$ of order $\ell$, then there 
exists an isogeny $\phi\from E\to E'$ with kernel $G$, and $\phi$ is uniquely 
determined up to isomorphism. In this case we write $E' = E/G$.

Call an isogeny $\phi$ \emph{$k$-rational} if it can be written as rational 
fractions with coefficients in $k$. One can show that $\phi$ is $k$-rational if 
and only if $\Ker(\phi)$ is.
Rationality is related to the Frobenius endomorphism of the curve $E$, which is 
defined as follows. Let $q$ be the order of $k$, and write $E$ as a plane curve 
with coordinates $x, y$. Then the Frobenius of $E$ is
\[
\begin{aligned}
\pi_E \from E &\to E \\
 (x, y) &\mapsto (x^q, y^q).
\end{aligned}
\]

A point on $E$ is $k$-rational if and only if it is invariant under $\pi_E$, 
and a subgroup of $E(\bar{k})$ is $k$-rational if and only if it is (globally) 
stable under $\pi_E$. More generally, being defined over an extension of $k$
of degree $r$ is the same as being invariant by the iterated Frobenius $\pi_E^r$.
Going back to the description of $E[\ell](\bar{k})$ as a 
vector space over $(\Z/\ell\Z)$, we are saying that an $\ell$-isogeny $\phi$ is 
$k$-rational if and only if the line $\Ker(\phi)$ is an eigenspace for the 
endomorphism $\pi_E$ restricted to $E[\ell](\bar{k})$. If $v$ denotes the 
eigenvalue of $\pi_E$ on this subspace, we will say that $\phi$ has \emph{
direction $v$}.


The number of rational isogenies from $E$ to another curve is then related to 
the behaviour of $\pi_E$ as an endomorphism of $E[\ell](\bar{k})$. If it acts 
as a scalar matrix, all $\ell+1$ lines are left stable, so $\ell+1$ rational 
isogenies exist; while if it acts as
\[
\left(
\begin{matrix}
\alpha & * \\
0 & \alpha
\end{matrix}
\right)
\quad
\text{(with $\alpha$ and $*$ nonzero in $\Z/\ell\Z$)}
\]
there is only one stable line, hence only one rational isogeny. In these cases, 
we say that $\pi_E$ has a \emph{double eigenvalue} modulo $\ell$, or that $\ell$
 is \emph{ramified}.

Two other situations can arise: either $\pi_E$ has two distinct eigenvalues 
modulo $\ell$ and there are two rational $\ell$-isogenies starting from $E$, or 
$\pi_E$ has no eigenvalue at all and there are no rational $\ell$-isogenies. 
Following the tradition dating back to the study of Schoof's algorithm for 
point counting, we will call such primes \emph{Elkies} and \emph{Atkin} 
respectively.

Given a line $L$ in $E[\ell](\bar{k})$ on which the Frobenius has eigenvalue $v$,
we can describe the field of definition of the points on $L$. The degree of this
field over $k$ is the smallest integer $r$ such that $\pi_E^r$ is the identity
on $L$: therefore $r$ is simply the multiplicative order of $r$ in $\Z/\ell\Z$.



\subsection{Ordinary elliptic curves and the theory of complex multiplication}

An elliptic curve $E/k$ is called \emph{ordinary} when its endomorphism ring $
\End(E)$ is an order in a quadratic imaginary field (other curves are called 
\emph{supersingular} and their endomorphism rings are orders in a quaternion 
algebra). Here, an \emph{order} is a subring which is a $\Z$-module of maximal 
rank. 

From now on we will restrict attention to ordinary curves and rational 
isogenies. For example, we will call two curves $E$ and $E'$ \emph{isogenous} 
if there exists a rational isogeny from $E$ to $E'$. This is an equivalence 
relation: according to a theorem of Tate, $E$ and $E'$ are isogenous if and 
only if they have the same number of points over $k$. Hasse's theory then 
implies that the rings $\Z[\pi_E]$ and $\Z[\pi_{E'}]$ are naturally isomorphic (
$\pi_{E'}$ being the image of $\pi_{E}$): indeed we have
\[
\pi_E^2 - t_E\pi_E + q = 0
\]
where $t$ is an integer such that
\[
\Card(E(k)) = p + 1 - t_E.
\]
These rings are of finite index in the endomorphism rings, and as a consequence 
$\End(E)$ and $\End(E')$ can be viewed as orders in the \emph{same} quadratic 
imaginary field $K = \Z[\pi_E]\otimes\Q$. Furthermore, these isomorphisms can 
be made coherent across an isogeny class in the sense that the Frobenius 
endomorphism always corresponds to the same element of $K$.

If $\phi\from E\to E'$ is an $\ell$-isogeny, then the orders $\End(E)$ and $\End
(E')$ are very close to each other. One has the following classification: either
\[
\begin{aligned}[l]
&\End(E) = \End(E'),
&\qquad\text{$\phi$ is then called \emph{horizontal}, or} &\\
&[\End(E):\End(E')] = \ell,
&\qquad\text{$\phi$ is then called \emph{descending}, or} &\\
&[\End(E'):\End(E)] = \ell,
&\qquad\text{$\phi$ is then called \emph{ascending}.} &
\end{aligned}
\]

We now present the group action used in the Rostovtsev--Stolbunov key exchange 
protocol. Let $\frak a$ be an ideal in $\End(E)$. Then we have a natural \emph
{${\frak a}$-torsion} subgroup of $E$ to look at:
\[
E[\frak a](\bar{k}) = \set{P\in E(\bar{k}) \suchthat \sigma(P) = 0\ 
\forall\sigma\in\End(E)}.
\]
This subgroup is the kernel of a rational isogeny $\phi_{\frak a}$ that is well-
defined up to postcomposition by an isomorphism, hence the codomain of $\phi_{
\frak a}$ is well-defined up to isomorphism over $\bar{k}$. We will call this 
codomain $\frak a\cdot E$, in other words
\[
\frak a\cdot E = E/E[\frak a].
\]

The isogeny $\phi_{\frak a}$ is always horizontal, hence we have $\End(\frak a
\cdot E) = \End(E)$, and its degree is the \emph{norm} of $\frak a$ as an ideal 
of $\End(E)$.
Let us call $\Ell_k(\O)$ the set of isomorphism classes over $\bar{k}$ of 
curves whose endomorphism ring is isomorphic to $\O$. It turns out that the 
construction above may be extended into a group action: namely, the group of 
fractional ideals of $\End(E)$ acts on the set $\Ell_k(\End(E))$. Furthermore, 
principal ideals act trivially, so that this action factorises as an action of 
the \emph{ideal class group} $\Cl(\End(E))$ on the set $\Ell_k(\End(E))$.

The main theorem of complex multiplication states that this action is \emph{
simply transitive}; in other terms, $\Ell_k(\End(E))$ is a principal 
homogeneous space under the group $\Cl(\End(E))$. If we fix a curve $E$ as a 
base point, we thus have a bijection
\[
\begin{aligned}
\Cl(\End(E)) &\to \Ell_k(\End(E)) \\
\text{Ideal class of }\frak a &\mapsto \text{Isomorphism class }\frak a\cdot E.
\end{aligned}
\]

Remember that the Frobenius characteristic equation
\[
X^2 - t_E X + q = 0
\]
of discriminant $\Delta_E$ is a defining equation for the field $K = \End(E)
\otimes\Q$. The discriminant of $\End(E)$ divides $\Delta_E$ since $\End(E)$ 
contains the ring $\Z[\pi_E]$. Hence, we can summarize the previous discussion 
in the following way: let $\ell$ be an odd prime, prime to the characteristic. 
Then either
\begin{enumerate}
\item $\Delta_E$ is not a square modulo $\ell$. Then $\ell$ is Atkin, and there 
are no rational $\ell$-isogenies starting from $E$. The prime $\ell$ is inert 
in $K$, so that there are no ideals of norm $\ell$ in $\End(E)$.
\item $\Delta_E$ is nonzero and a square modulo $\ell$. Then $\ell$ is Elkies, 
and there are exactly two $\ell$-isogenies starting from $E$ whose directions 
are the two Frobenius eigenvalues modulo $\ell$. They are horizontal since $\ell
$ is prime to $\Delta_E$. The prime $\ell$ splits in $K$, and there are two 
ideals of norm $\ell$ in $\End(E)$ that are complex conjugates of each other, $
\frak a_\ell$ and $\frak a_\ell^{-1}$. One can show that the action of the 
ideal $\frak a_\ell$ in always given by $\ell$-isogenies in the same direction $
v$, which satisfies
\[
\pi_E = v \mod \frak a_\ell.
\]
Of course, the same is true for $\frak a_\ell^{-1}$. Since the eigenvalues are 
distinct, we may safely identify the ideal $\frak a_\ell$ with the pair $(\ell, 
v)$.
\item $\Delta_E$ is zero modulo $\ell$. Then $\ell$ is ramified. The situation 
is more complicated: there can be either zero or one ideal of norm $\ell$ in $
\End(E)$, depending on its conductor, and there may be non-horizontal 
isogenies. In any case, all these isogenies are associated with the same 
Frobenius eigenvalue.

\end{enumerate}


\subsection{Modular curves}

Elliptic curves equipped with $\ell$-isogenies (or, more generally, some 
additional structure) can be naturally identified to \emph{points} on a 
mathematical object called a \emph{modular curve}. Constructing modular curves 
in general is not easy, and we will simply recall here some well-known 
properties. As above, let $\ell$ be a prime distinct from the characteristic of 
$k$. 
Then there exists a smooth curve $X_0(\ell)$ defined over $k$ with the 
following property: there is a bijection
\[
X_0(\ell)(k) \isom \set{
\begin{matrix}
\text{Isomorphism classes over $\bar{k}$ of pairs $(E, G)$}\\
\text{where $G$ is a subgroup of order $\ell$ defined over $k$.}
\end{matrix}
}
\]

Of course, this is not a mere bijection: it is functorial in $k$, and one side 
can be deduced from the other by algebraic means.
In this work, we use modular curves presented as \emph{plane curves} up to 
birational equivalence, i.e. given as an equation of the form
$\Phi(X, Y) = 0.$
In order to find such an equation, it is enough to find to functions $X$ and $Y$
 on the curve that generate its function field. One can obtain an equation for $
X_0(\ell)$ using the two functions
\[
\begin{aligned}
X(E, G) &= j(E), \\
Y(E, G) &= j(E/G)
\end{aligned}
\]
where $j(E)$ denotes the $j$-invariant of the elliptic curve $E$. Indeed, $X$ 
and $Y$ are well defined and algebraic, and can be shown to generate the 
function field of $X_0(\ell)$. They are linked by a polynomial equation
\[
\Phi_\ell(X, Y) = 0
\]
where $\Phi_\ell$ is called the \emph{classical modular polynomial} of level $
\ell$. The coefficients of $\Phi_\ell$ can be obtained by considering the 
modular curve $\Phi_\ell$ over $\C$, for example.

These polynomials quickly become difficult to compute, hence one uses different 
functions on $X_0(\ell)$ when $\ell$ becomes large, giving rise to the so-
called \emph{Atkin modular polynomials}. The classical polynomial $\Phi_\ell$ 
is still arguably the simplest to use, since one only has to solve the 
polynomial equation $\Phi_\ell(j(E), Y) = 0$ to find the $j$-invariants of 
curves linked to $E$ by an $\ell$-isogeny. Here one sees how equations of 
modular curves can be used in the context of isogeny computations.

In general, the curve $\Phi(X, Y) = 0$ is no longer smooth: for example, double 
points may appear, so that $k$-points on this curve are no longer in bijection 
with geometric data as above. Still, plane equations are useful. For example, 
one can show that double points of $X_0(\ell)$ in the coordinates $X, Y$ 
correspond to elliptic curves $E$ that have a nontrivial endomorphism of degree 
$\ell^2$, and this is easily controlled: in particular it cannot be the case
when the discriminant of $\End(E)$ is greater than $\ell^2$.


\section{The Rostovtsev--Stolbunov key-exchange protocol}
\label{sec:keyex}

\subsection{Key exchange from abelian Cayley graphs}

The Rostovtsev--Stolbunov key-exchange protocol is an instance of a general
framework of key exchanges using abelian Cayley graphs. This general pattern
is in turn an example of Couveignes' `Hard Homogeneous Spaces' key echange
framework. Let us first describe this general
setting before discussing more specific algorithms used in isogeny computations.
The public data is
\begin{itemize}
\item A finite abelian group $G$;
\item A finite set $S$ of elements of $G$;
\item A finite set $X$ on which $G$ acts simply transitively;
\item A fixed element $x_0\in X$;
\item For each $s\in S$, an integer $M_s\geq 1$.
\end{itemize}
We further ask that the action of $G$ on $X$ be \emph{computable}, in the sense
 that there is an algorithm \algnamestyle{Step} which, given $s\in S$ and $x\in X$ as
input, computes the element $s\cdot x$. To explain the name, let $\Graph(G, S, X)$
be the oriented graph whose vertices are the elements of $X$, and an edge labelled
by $s\in S$ links $x_1$ to $x_2$ if $s\cdot x_1 = x_2$; then the algorithm
\algnamestyle{Step} allows us to follow one edge in the graph. The graph $\Graph(G, S, X)$
may be (at least theoretically, if not computationally) identified with the usual
Cayley graph associated with the abelian group $G$ and the set $S$ of generators.

As usual, two participants Alice and Bob want to build a common secret while
communicating over an insecure channel. Using the public data above, they may
achieve this by computing several walks in the graph $\Graph(G, S, X)$. This is
described by the algorithms \ref{alg:keyex}, \ref{alg:key} and \ref{alg:walk}.

\begin{algorithm}
    \caption{\algnamestyle{KeyExchange}: key exchange using an abelian Cayley graph}
    \label{alg:keyex}
    \KwIn{Public data as above}
    \KwOut{A secret $x_S\in X$ shared by Alice and Bob}
    %		
    Alice does\\
			\quad $K_A \gets$ \algnamestyle{KeySample}()\;
			\quad $x_A \gets$ \algnamestyle{Walk}$(K_A, x_0)$\;
			\quad \textbf{publish} $x_A$ \;
		Bob does\\
			\quad $K_B \gets$ \algnamestyle{KeySample}()\;
			\quad $x_B \gets$ \algnamestyle{Walk}$(K_B, x_0)$\;
			\quad \textbf{publish} $x_B$ \;
		Alice does\\
			\quad $x_{S, A}$ = \algnamestyle{Walk}$(K_A, x_B)$\;
		Bob does\\
			\quad $x_{S, B}$ = \algnamestyle{Walk}$(K_B, x_A)$\;
		\Return $x_S = x_{S, A} = x_{S, B}$
\end{algorithm}

\begin{algorithm}
	\caption{\algnamestyle{KeySample}: construction of an ephemeral key}
	\label{alg:key}
	\KwIn{Public data as above}
	\KwOut{An ephemeral key}
	%
	\For{$s\in S$}{
		$k_s \overset{R}{\gets} [0, M_s]$ \quad (uniformly at random)
	}
	\Return{$(k_s)_{s\in S}$}
\end{algorithm}

\begin{algorithm}
	\caption{\algnamestyle{Walk}: a walk in the abelian Cayley graph}
	\label{alg:walk}
	\KwIn{Public data as above; an ephemeral key $K = (k_s)_{s\in S}$; a point $x_I\in X$}
	\KwOut{The element $x_F = (\prod_{s\in S} s^{k_s})\cdot x_I$}
	%
	$x_F\gets x_I$\;
	\For{$s\in S$}{
		\lFor{\(0 \le i < k_s\)}{
			$x_F\gets$ \algnamestyle{Step}$(s, x_F)$
		}
	}
	\Return $x_F$
\end{algorithm}

As the reader may see, this key exchange framework is based on the computation
of several walks in the graph $\Graph(G, S, X)$. Alice and Bob are allowed to
compute steps in the directions specified by the elements of $S$, and the integers
$M_s\geq 1$ play the role of the maximal number of steps that will be computed
for each generator. Of course, when both $s$ and $s^{-1}$ appear in $S$, one
of $k_s$ and $k_{s^{-1}}$ has to be zero, since the steps would otherwise
annihilate each other.
An important remark is that there is no need to represent
elements of $G$ at all in this protocol, or to compute with them directly,
outside the algorithm \algnamestyle{Step}.

When designing practical instances of abelian Cayley graph
protocols, the main degrees of liberty are the choice of the generators themselves,
and the bounds $M_s$. The cost of the key exchange is a weighted
sum of the $M_s$, while the key space size is the product of these integers. When
the number of generators is sufficient, there is therefore an exponential gap
between the key space size and the cost of the walks.

Of course, in order to be
cryptographically interesting, the action of the group $G$ on $X$ must in particular
be difficult to \emph{invert}: $x_1, x_2\in X$ being given, finding a combination
$g\in G$ of the generators such that $g\cdot x_1 = x_2$. must be a hard problem.
This is the main reason why this framework cannot usually be used with
Cayley graphs given by multiplication in the group, since division in $G$ is usually
a simple operation. To be more precise, the security of this key exchange relies
on the difficulty of the following Diffie--Hellman-like problems:

\begin{prob}[Group Action Decisional Diffie--Hellman (GADDH) problem] Let $x, y\in X$,
$a, b\in G$. Given $x,\ y,\ a\cdot x$ and $b\cdot x$, decide whether $(ab)\cdot x = y$.
\end{prob}

\begin{prob}[Group Action Computational Diffie--Hellman (GACDH) problem] Let $x\in X$,
$a, b\in G$. Given $x,\ a\cdot x$ and $b\cdot x$, compute $(ab)\cdot x$.
\end{prob}

For a particular family of group actions, we will call GADDH assumption the assumption
that the GADDH problem is computationally infeasible, in the sence that the advantage
given by any polynomial-time algorithm is a negligible function of the security parameter
$\log|G|$.

\subsection{Presentation of the protocol}

The idea of Rostovtsev and Stolbunov is to instanciate the general framework of
key exchange based on abelian group actions using isogenies between ordinary elliptic
curves. Let us fix an ordinary elliptic curve $E_0$ over a finite field $k$, with
endomorphism ring $\O$. Then the ideal class group of $\O$ will play the role of
the abelian group $G$, and the set $X$ is the set $\Ell_k(\O)$ of elliptic curves
with endomorphism ring $\O$.

We also have to describe the set $S$ of generators used. Let $L$ be a list of
small Elkies primes for the curve $E_0$, as defined in section \ref{sec:math}.
Then for each $\ell\in L$, there exists two ideals in $\O$ of norm $\ell$, which
are inverses of each other. These ideal classes will be the elements of $S$. According to
the discussion in section \ref{sec:math}, we may identify these ideal classes with
pairs $(\ell, v)$, where $v$ is one of the two Frobenius eigenvalues modulo $\ell$.

Now the description of the algorithm \algnamestyle{Step} (algorithm \ref{alg:step}) is a direct consequence of
the previous discussions. The subroutine \algnamestyle{IsogenyKernel}$(E, \ell, j_1)$
is an algorithm which computes the kernel of an isogeny from $E$ to an elliptic curve
with $j$-invariant $j_1$, and that we will discuss in the following section.
As before, we identify isomorphism classes of curves with
$j$-invariants.

\begin{algorithm}
	\caption{\algnamestyle{Step}: a step in an ordinaty isogeny graph}
	\label{alg:step}
	\KwIn{A generator $(\ell, v)$; an isomorphism class $j_I\in \Ell_k(\O)$; a curve $E_I$
		such that $j(E_I) = j_I$; the classical
		modular polynomial $\Phi_\ell(X, Y)$}
	\KwOut{The isomorphism class $j_F$ given by the action of $(\ell, v)$ on $j_I$}
	%
  $P\gets \Phi_\ell(j_I, Y)$\;
	$j_1, j_2 \gets$ \algnamestyle{Roots}$(P, k)$\;
	$K_1\gets $ \algnamestyle{IsogenyKernel}$(E_I, \ell, j_1)$\;
	\lIf{\algnamestyle{HasFrobeniusEigenvalue}$(K_1, v)$}{\Return $j_1$}
	\lElse{\Return $j_2$}
\end{algorithm}

According to section \ref{sec:math}, this algorithm is valid if the discriminant
of $\O$ is larger than $\ell^2$.
Note that in algorithm \ref{alg:step}, there is no need to know the endomorphism
ring $\O$, to compute directly with ideals or to be able to compute the set
$\Ell_k(\O)$. In order to use this
protocol, one only has to know which primes are Elkies for the curve $E_0$,
and this is easy once we know the Frobenius characteristic equation on $E_0$.
This is equivalent to computing the cardinality of $E_0$, which is
done in polynomial time using Schoof's algorithm. If we replace $v$ with the other
Frobenius eigenvalue modulo $\ell$ in algorithm \ref{alg:step}, we will compute
a step in the direction specified by the other ideal of norm $\ell$: therefore
we are able to use both of them as generators.

According to section \ref{sec:math}, we may give another version for the
algorithm \algnamestyle{Step}, described in algorithm \ref{alg:steptors}.

\begin{algorithm}
	\caption{\algnamestyle{Step2}: another way of computing a step in the isogeny graph}
	\label{alg:steptors}
	\KwIn{A generator $(\ell, v)$; an isomorphism class $j_I\in \Ell_k(\O)$; a curve $E_I$
		such that $j(E_I) = j_I$; the classical
		modular polynomial $\Phi_\ell(X, Y)$}
	\KwOut{The isomorphism class $j_F$ given by the action of $(\ell, v)$ on $j_I$}
	%
	$r\gets$ \algnamestyle{MultiplicativeOrder}$(v, \F_\ell)$\;
	$k_{ext}\gets$ \algnamestyle{Extension}$(k, r)$\;
	$P_{ext}\gets$ \algnamestyle{TorsionPoint}$(E_I, \ell, k_{ext})$\;
	$K\gets$ \algnamestyle{Subgroup}$(P_{ext})$\;
	$E_F\gets$ \algnamestyle{Quotient}$(E_I, K)$\;
	\Return $j(E_F)$
\end{algorithm}

Algorithm \ref{alg:steptors} is valid when the Frobenius eigenspace with eigenvalue $v$ is
the only line in $E[\ell](\bar{k})$ whose points are defined over the degree $r$ extension
$k_{ext}$ of $k$. In other words, it is valid when the multiplicative order of $v$ in
$\F_\ell$ is strictly smaller than the multiplicative order of the second eigenvalue.
In any case, only one of the two ideals of norm $\ell$ can be eligible for algorithm
\ref{alg:steptors}.

Let us now describe the primitives about isogeny computations used in algorithms
\ref{alg:step} and \ref{alg:steptors}. This will allow us to compare the merits
of these two algorithms, which is directly related to the choice of the public
parameters.


\section{Algorithms and parameter selection}

We keep the notations of sections \ref{sec:math}
and \ref{sec:keyex}. In order to analyse the costs of various parts of the
algorithm, we make some complexity estimates in terms of operations in the
base field $k$, where $\ell$ and $\log q$ are thought of as variables.
We will use the soft-$O$ notation: a cost $\softO(f(\ell, \log q))$ means
$O(f(\ell, \log q)^{O(1)})$. These complexity estimates will also
allow us to discuss the relative costs of algorithms \ref{alg:step} and
\ref{alg:steptors}, and guide us in the later choice of the primes $\ell$
and bounds $M_\ell$ for the number of steps that we will finally propose.

\subsection{Description of the subroutines}


The reader will agree that the purpose of the subroutines \algnamestyle{Roots}
and \algnamestyle{MultiplicativeOrder} is clear. Remember that the
prime $\ell$ is very small compared to the order of $k$. Therefore we use the
well-known Cantor--Zassenhaus method in \algnamestyle{Roots}:
given a polynomial $P$,
we compute $F = X^q \mod P$ using a fast exponentiation method, and then the polynomial
$\gcd(F - X, P)$ is the product of irreducible factors of degree one in $P$
in $k[X]$. In our setting the modular equation has two roots: therefore we only need
to compute square roots to complete the \algnamestyle{Roots} algorithm, which is
done using the Tonnelli--Shanks algorithm. The cost is dominated
by that of computing $F$: when $P$ is a modular equation
of level $\ell$, hence degree $\ell + 1$, this costs $\softO(\ell\log q)$ operations in $k$
using asymptotically fast methods for polynomial arithmetic. As for \algnamestyle{MultiplicativeOrder},
any version of it will do, including the really naive one consisting in computing
all the powers of $v$ until one is equal to 1.

Algorithm \algnamestyle{HasFrobeniusEigenvalue}
involves computing the Frobenius on the kernel: more precisely, one computes $(X^q, Y^q)$
where $q$ is the order of $k$ in the quotient ring
\[
k[X, Y]/(K(X),\ Y^2 - X^3 - aX - b)
\]
where $a, b$ are given by the curve equation $y^2 = x^3 + ax + b$. Here, the polynomial
$K(X)$ is the degree $\frac{\ell-1}{2}$ polynomial whose roots are the $x$-coordinates of the elements
in the subgroup of order $\ell$. One then checks whether it is equal to $v$ times the generic
point $(X, Y)$ which is computed using the addition law on the curve. In fact, computing
in the double quotient ring above is not even needed. The two quantities we want to compute,
$(X^q, Y^q)$ and $[v](X, Y)$, have a $Y$ factor only in their second coordinate. Thus one
can only store univariate polynomials in $X$ and multiply by the square of $Y$, that is
$X^3 + aX + b$, at appropriate places in the addition algorithm on the curve. Thus the
computations are really done in the ring $k[X]/K(X)$.

There is one last subroutine in algorithm \ref{alg:step} that is important to discuss,
namely the evaluation (and the storage) of the bivariate polynomial $\Phi_\ell$. It is a bivariate
polynomial of degree $\ell + 1$ in both variables, which is not sparse and whose coefficients
grow quickly. Therefore in first approximation, even its storage could cost as much as
$\O(\ell^2 \log q)$ when the coefficients are reduced in $k$. However, it is in practice a negligible
part of the algorithm: when the classical modular polynomials become too cumbersome to store, we
switch to Atkin modular polynomials. If one wants to use modular equations of yet higher levels,
one can compute directly the classical modular polynomial evaluated at an element in $k$,
with a cost that is linear in the level. However, using such methods is not mandatory in
the context of the Rostovtsev--Stolbunov key exchange.

\subsection{Isogeny computations}

The algorithms \algnamestyle{IsogenyKernel} and \algnamestyle{Quotient} appearing in
algorithms \ref{alg:step} and \ref{alg:steptors} are well-known primitives in isogeny
computations. Computing a quotient is easy, whereas the problem of computing the
kernel of an isogeny given its domain, degree and image is a somewhat more difficult
problem. A rich literature exists on this subject, so that we will content ourselves
with the larger picture.

The \algnamestyle{Quotient} algorithm directly uses Vélu's formulas, which are
summarised in the following theorem.
\begin{theorem}[Vélu]

\end{theorem}

-------

faire un review rapide des
algorithmes à notre disposition pour le calcul d'isogénies, en faisant
référence aux articles pertinents.

\subsection{Cost analysis}

Faire une première remarque qu'il est plus efficace d'utiliser des
courbes qui ont plus de petits nombres premiers d'Elkies.

Analyse du coût de l'algorithme \ref{alg:steptors}.

Cela amène à vouloir chercher des courbes dont la trace mod l a de
bonnes propriétés pour de petits l.


\subsection{Looking for a good initial curve}

Présenter la stratégie de
recherche SEA--early-abort. Cela vaut-il la peine de parler de la
sélection de candidats à l'aide d'une courbe modulaire, vu le peu que
cela fait gagner à la fin ? Je trouve que c'est intéressant, mais
l'intérêt est plutôt théorique je dirais.

\section{Experimental results}

On présente des paramètres pour viser à 128 bits de sécurité
classique.  Donner le corps de base choisi, et la meilleure courbe
obtenue par la recherche ci-dessus, la trace du Frobenius et le
discriminant de l'ordre maximal (qu'il faudra calculer avec ECM par
exemple, car on peut espérer qu'il n'a pas deux grands facteurs
premiers...), et une minoration explicite du nombre de classes.
Proposer aussi le nombre de pas à effectuer pour chaque nombre premier
d'Elkies.

Indiquer l'existence du paquet Julia avec un exemple complet d'échange
de clés. dire: tant de temps a été dépensé en
calculs de Frobenius, tant de temps en multiplications scalaires, etc.

Faire aussi des commentaires sur la sécurité quantique de ces
paramètres, en fonction de la discussion précédente. 

\section{Security}

Deux parties: d'abord la sécurité classique, ensuite la sécurité quantique.

Au niveau classique, on indique l'algorithme de Galbraith (\& Hess \&
Smart, ou \& Stolbunov...). Peut-être mentionner le fait que la
sécurité repose sur des heuristiques, à savoir une bonne répartition
des marches aléatoires dans le graphe, à des niveaux très inférieurs à
ce que peut garantir GRH.  Il reste des questions non résolues: quelle
est l'impact de la structure de groupe ? (Dans ses slides Stolbunov
insiste beaucoup pour avoir un groupe cyclique, je n'en vois pas
vraiment l'intérêt si ce n'est une forme de superstition !)  Est-ce
que la structure supplémentaire des courbes peut être exploitée ?

Au niveau quantique: l'attaque est certes sous-exponentielle, mais on
peut tout de même tenter d'évaluer des paramètres de sécurité. On peut
peut-être s'inspirer des recommandations du NIST ? Quelles ressources
demande l'attaque de Childs et al. relativement à ce que le NIST
envisage ?


\section{Conclusion}

- A quel point a-t-on accéléré le cryptosystème ?
- Ce protocole a-t-il finalement un intérêt dans le cadre de la crypto post-quantique ?


\end{document}
